%--------1---------2---------3---------4---------5---------6---------7---------8---------9---------1---------2---------3---------4---------5---------6
%23456789 123456789 123456789 123456789 123456789 123456789 123456789 123456789 123456789 123456789 123456789 123456789 123456789 123456789 123456789

\glsunsetall
\chapter{Caveats about using \texorpdfstring{\gls{MSE}}{MSE} for \texorpdfstring{\gls{RP-MS}}{RP-MS}}
\label{chap:MSE}
\glsresetall

\begingroup
\thinmuskip=0mu
\medmuskip=0mu
\thickmuskip=0mu

One run of targeted \gls{LC-MS/MS} usually can cover only one peptide.
However, one run of LC-\gls{MSE} covers all peptides in the sample analyzed by this run.
Thus, we hypothesized that \gls{MSE} can improve the spatial resolution of \gls{RP-MS},
		because \gls{MSE} could make \gls{RP-MS} at subpeptide resolution less labor-intensive and less time-consuming if our hypothesis is true.{}
Unfortunately, our hypothesis is wrong.
However, we learned some important lessons that can be shared.
This chapter
		presents the background on \gls{MSE},
		presents an \gls{MSE} dataset,
		presents a lower bound on the interference to desired signal in \gls{MSE} spectra,
		presents how the \gls{MSE} dataset failed to confirm our hypothesis,
		and finally presents why our hypothesis is wrong.
Past works related to \gls{RP-MS} are presented in \cref{sec:oxlvl:relatedworks} and are thus omitted in this chapter.

\section{Background of \texorpdfstring{\gls{MSE}}{MSE}}

\Gls{MSE}, a technology in mass spectrometry, is pioneered by Waters Corporation \cite{plumb2006uplc}.
The superscripted letter E in \gls{MSE} stands for varying levels of energy.
In \gls{MSE}, the \gls{CID} alternates between low-energy mode and high-energy mode.
In low-energy mode, collision energy is low. 
Thus, the percentage of precursor ions that fragment and then can respectively become product ions is low. 
Thus, low-energy \gls{CID} produces \gls{MS1}-like spectra.
In high-energy mode, collision energy is high.
Thus, the percentage of precursor ions that fragment and then can respectively become product ions is high. 
Thus, high-energy \gls{CID} produces \gls{MS2}-like spectra.
In \gls{MSE}, all molecules coming from the inlet of a mass spectrometer are selected for fragmentation regardless of \gls{CID} mode.
Thus, the precursor selectivity in \gls{MSE} is low.

\Gls{MSE} has several advantages compared with \gls{MS/MS}.
First, \gls{MS/MS} selects at each \gls{RT} only the precursors that satisfy certain predefined conditions for \gls{MS2}.
This satisfaction is highly non-reproducible.
However, \gls{MSE} selects at each \gls{RT} all precursors for high-energy \gls{CID}.
Thus, precursor selectivity is relative constant across \gls{MSE} experiments but varies across \gls{MS/MS} experiments.
Thus, the result generated by \gls{MSE} is generally more reproducible than the result generated by \gls{MS/MS}.
Moreover, \gls{MS/MS} selects at each \gls{RT} only the precursors that are within a narrow window of \gls{m/z} for \gls{MS2}.
However, \gls{MSE} selects at each \gls{RT} all precursors for high-energy \gls{CID}.
Thus, the analysis by \gls{MSE} is more comprehensive than the analysis by \gls{MS/MS}.

Unfortunately, given the same precursors to be selected for either high-energy \gls{CID} or \gls{MS2},
	\gls{MSE} selects a larger quantity of more-chemically-heterogeneous precursors than \gls{MS/MS} would select.
Thus, \gls{MS2}-like spectra produced by \gls{MSE} are both more complex and noisier than \gls{MS2} spectra produced by \gls{MS/MS}.
	
Mass spectra produced by one run of \gls{MSE} can identify endogenous metabolites in rat urines \cite{plumb2006uplc}.
Moreover, appropriate processing of \gls{MSE} spectra can enhance the discovery of metabolites \cite{bateman2007mse}.
Unfortunately, \gls{MSE} has been used for only characterizing small metabolites.
Thus, performance of \gls{MSE} for protein \glsfirst{MS} is unknown, and performance of \gls{MSE} for \gls{RP-MS} is completely unknown.
However, one run of \gls{MSE} can potentially cover all peptides that would require multiple runs of conventional \gls{MS/MS} to cover.
Thus, evaluating the performance of \gls{MSE} for studying proteins is important.
For example, to cover a peptide by \gls{LC-MS/MS}, one run has to target this peptide during the entire \gls{RT} range of this peptide.
Thus, the coverage of multiple peptides of interest requires multiple such runs of \gls{LC-MS/MS}.
However, if the \gls{MS2}-like spectra produced by \gls{MSE} are not much noisier and not much more complex than the \gls{MS2} spectra produced by \gls{MS/MS}, then the coverage of multiple peptides of interest would require only one run of \gls{MSE}.  
Moreover, \gls{MSE} selects all precursors for high-energy \gls{CID}.
Thus, \gls{MSE} can potentially cover all oxidized products of this peptide of interest in only one run.
		
\section{The \texorpdfstring{\gls{MSE}}{MSE} dataset}
\begin{figure}
\includegraphics[trim=0 0 0 0, width=\textwidth]{img/0184_200_TEAEMK.pdf}
\caption[
	A pair of consecutive mass spectra in the \gls{MSE} dataset.]{
	A pair of consecutive mass spectra in the \gls{MSE} dataset.
	First, the low-energy \gls{CID} of \texttt{TEAE(M)(+15.99)K} generated the upper \gls{MS1}-like spectrum.
	Immediately afterwards, the high-energy \gls{CID} of \texttt{TEAE(M)(+15.99)K} generated the lower \gls{MS2}-like spectrum. 
	The y-axis represents relative intensity.
	\label{fig:rpmse:dataset:TEAEMK196}
}
\end{figure}
	
%\Gls{MSE} is less labor-intensive and less time-consuming than \gls{MS/MS}.
%Thus, we hypothesized that \gls{MSE} can still improve the spatial resolution of \gls{RP-MS}.
%If our hypothesis is true, then \gls{MSE} can make \gls{RP-MS} less labor-intensive and less time-consuming.
%We hypothesized that \gls{MSE} can still improve the spatial resolution of \gls{RP-MS}.
%To verify our hypothesis,	we analyzed the \gls{MSE} dataset.
The \gls{MSE} dataset is generated by an \gls{RP-MS} experiment conducted by Siavash Vahidi and Professor Lars Konermann.
The mass spectrometer performed \gls{MSE} instead of \gls{MS/MS} in this \gls{RP-MS} experiment that is otherwise standard. 
This \gls{RP-MS} experiment proceeded as follows.
First, \gls{FPOP} was performed on apomyoglobin (PDB \texttt{1WLA}).
After this \gls{FPOP}, some apomyoglobins were covalently modified. 
%while some others were not covalently modified.
Next, trypsin cleaved all apomyoglobins into peptides.{}	
Then, these peptides were eluted and thus separated by \gls{HPLC}.
While \gls{HPLC} is eluting these peptides, the peptides that \gls{HPLC} finished eluting were ionized by, analyzed by, and then detected by a Synapt G2 mass spectrometer (Waters, Milford, MA).
This mass spectrometer was always in \gls{MSE} mode.
The \gls{CID} energy inside this mass spectrometer was alternating between \(20.0\si{\eV}\) and \(30.0\si{\eV}\).	
Finally, a sequence of raw \gls{MSE} spectra was generated by this mass spectrometer.
We converted this sequence of raw \gls{MSE} spectra into the mzML format by using MSConvert \cite{chambers2012cross}. 	

\section{A lower bound on the interference-to-signal ratios in the \texorpdfstring{\gls{MSE}}{MSE} dataset}

Let us define the following. 
\begin{enumerate}[nolistsep]
\item Let \({\Delta m}\) be the resolution of the mass spectrometer.
\item Let \(M\) be the length of the continuous \gls{m/z} range of the mass spectrometer such that almost all peak intensities are within this range.
\item Let \(n+1\) be the number of residues in a peptide of interest.
\item Let \(\mathnormal{r}\) be the following proportion in an \gls{MS1}-like spectrum: 
						the sum of the intensities respectively generated by ions of interest to the sum of the intensities respectively generated by all ions.
\item Let signaling peaks be the peaks that we are interested in.
      Let signal be the sum of the respective intensities of all signaling peaks.
      Let noisy peaks be the peaks that we are not interested in.
      Let noise be the sum of the respective intensities of all noisy peaks.
      Let an interfering peak be a noisy peak whose \gls{m/z} overlaps with the \gls{m/z} of any signaling peak.
      Let interference be the sum of the respective intensities of all interfering peaks.  
\end{enumerate}
Let us make the following optimistic assumptions.
\begin{enumerate}[nolistsep]
\item The respective \gls{m/z} of noisy peaks are evenly distributed in a range of length \(M\).
\item Every precursor ion forms at most one singly charged y-ion. This y-ion always has two isotopes.
\item The accuracy of the mass spectrometer is perfect.
\end{enumerate}
In high-energy \gls{CID} mode, \(\mathnormal{r}\) is approximately 
		the proportion of the sum of the intensities respectively generated by some product ions of interest 
		               to the sum of the intensities respectively generated by all ions.
Thus, for both \gls{MS1}-like spectra and \gls{MS2}-like spectra, \(\mathnormal{r}\) denotes the ratio of signal to noise.	
In \gls{RP-MS}, signaling peaks are respectively generated by only oxidized or unoxidized y-ions. 
These y-ions are respectively formed by only \gls{mono-oxidized} precursors.		
For each \(i{\in}[1{\dots}n]\), \(\texttt{y}_i\) can be either unoxidized or \gls{mono-oxidized} and generates two isotopic peaks.{}
Thus, \(\texttt{y}_i\) can generate \(2{\times}2{\times}n\) signaling peaks.
Thus, the signal is within a noncontiguous \gls{m/z} range of \(2{\times}2{\times}n{\times}\Delta{}m\),
	because each signaling peak has a width of \(\Delta{}m\).
Thus, \(\displaystyle \frac{1-\mathnormal{r}}{M} \div \frac{\mathnormal{r}}{2 \times 2 \times n \times \Delta{}m}\) 
		denote the ratio of interference to signal,
	because all noisy peaks are distributed over an \gls{m/z} range of length \(M\).

The following is observed in the \gls{MSE} dataset. 
\(M\approx1000\si{\dalton}\) because almost all peaks are in the \gls{m/z} range from \(100\si{\dalton}\) to \(1100\si{\dalton}\). 
\(\Delta m\approx0.1\si{\dalton}\). % by applying the definition of mass resolution to observed mass spectra.
\(n=10\) for a typical tryptic peptide of apomyoglobin (\gls{PDB} \texttt{1WLA}).
\(\mathnormal{r}\approx0.01\) for most precursors of interest,
	although the respective \(\mathnormal{r}\) of two precursors can differ by orders of magnitude.
Thus, 
	\(\displaystyle
		\frac{1-\mathnormal{r}}{M} % 0.99 / 1000
		\div
	  \frac{\mathnormal{r}}{2 \times 2 \times n  \times \Delta{}m} % 0.01 /4 
		\approx \frac{2}{5}\).	
Thus, on average, the interference is \(40\%\) of the signal.
 
Worse still, our assumptions are overly optimistic. 
For example,
	high-energy \gls{CID} can generate an ion that is not a standard y-ion,
	sources other than irrelevant precursor ions can generate noisy peaks,
	and the accuracy of the mass spectrometer is not perfect.
Thus, 
	\(y_i'\div(y_i+y_i')\) as a function of \(i\) is unlikely to be generally increasing,
	where \(y_i\) and \(y_i'\) are defined in \cref{sec:oxlvl:methods}.
 	 	
\section{Negative results on the \texorpdfstring{\gls{MSE}}{MSE} dataset}	

We attempted to reduce interference and to amplify signal.
Unfortunately, even our optimistic assumptions imply that the ratio of interference to signal is at least \(40\%\).
In reality, we observed that, for almost all signaling peaks, the ratio of interference to signal is much higher than this lower bound of \(40\%\). 
Moreover, different y-ions respectively generated by different precursors sometimes overlap with each other in both \gls{m/z} and \gls{RT}.
We observed that, as \(i\) increases, \((y_i')\div(y_i+y_i')\) randomly fluctuates instead of generally increasing,
	presumably because \(y_i\) and/or \(y_i'\) are subject to too much interference.
%\Cref{fig:rpmse:dataset:TEAEMK196} shows one of the best pairs of consecutive \gls{MSE} spectra found in the \gls{MSE} dataset.
The \gls{MS2}-like spectrum in \cref{fig:rpmse:dataset:TEAEMK196} is one of the best-quality \gls{MS2}-like spectra in the \gls{MSE} dataset.
Still, in the \gls{MS2}-like spectrum in \cref{fig:rpmse:dataset:TEAEMK196}, only \(\texttt{y}_1\),  \(\texttt{y}_2'\), and \(\texttt{y}_6'\) 
		can be detected by meticulous and labor-intensive visual inspection after zooming into the respective \gls{m/z} of these y-ions.

\section{Discussion about the negative results}	

\Gls{RP-MS} that uses \gls{MSE} is both less time-consuming and less labor-intensive than \gls{RP-MS} that uses targeted \gls{MS/MS}.
Thus, we attempted to use \gls{MSE} for improving the spatial resolution of \gls{RP-MS}.
Unfortunately, our attempt failed, presumably because the \gls{MS2}-like spectra produced by \gls{MSE} have too much noise-induced interference.
By making several optimistic assumptions, we established a lower bound on this interference.
Methods for reducing this interference may exist.
Still, we suspect that current \gls{MSE} technology cannot reliably quantitate most product ions generated by high-energy \gls{CID}.

The additional dataset described in \cite{muntel2014comprehensive} is also generated by a Synapt G2 mass spectrometer that runs in \gls{MSE} mode.
Thus, we looked at this additional dataset.
The experiment that generated this additional dataset has the following characteristics compared with the \gls{MSE} dataset.
First, the duration of each scan is increased to produce mass spectra with higher mass resolution.
Second, peptides are eluted more slowly to better separate these peptides.
Third, \gls{CID} seems to be more optimized.
Presumably because of these characteristics, we can manually identify some product ions in this additional dataset.
Unfortunately, the \gls{MS2}-like spectra in this additional dataset are generally still too noisy and too complex. 
Thus, we can manually quantitate only very few product ions of interest in this dataset.
Thus, \gls{MSE} currently seems to be unable to improve the spatial resolution of \gls{RP-MS}.

\endgroup