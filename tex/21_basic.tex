%--------1---------2---------3---------4---------5---------6---------7---------8---------9---------1---------2---------3---------4---------5---------6
%23456789 123456789 123456789 123456789 123456789 123456789 123456789 123456789 123456789 123456789 123456789 123456789 123456789 123456789 123456789

\glsunsetall
\chapter{Fundamentals of protein folding}
\label{chap:fund1}
\glsresetall

%Every biological organism is partially made of some proteins.

A peptide is a sequence of \(n\) amino-acid residues chained together by \(n{-}1\) peptide bonds, where \(n{\ge}2\).
In this thesis, ``residue'' refers to only ``amino-acid residue'' unless explicitly stated otherwise.
A polypeptide is a relatively long peptide.	
Every polypeptide is composed of almost only the 20 standard amino-acid residues.{} 
Different sequences of these 20 residues correspond to different polypeptides.
Thus, a lot of different polypeptides can exist, because there are \(20^n\) distinct sequences that have a length of \(n\).
Although the number of polypeptides in a typical biological organism is much less than \(20^n\), this number is still huge.
A protein is an assembly of at least one polypeptide. 
Almost every protein is found in at least one biological organism.
Almost every protein catalyzes a metabolic reaction that is often essential for the survival of at least one biological organism. 
The metabolism of every biological organism requires numerous proteins.

The structure of a protein is strongly correlated with the function of this protein. 
Thus, determination of protein structure is a fundamental problem in life science. 
Protein structure can be observed at different resolutions.
Four levels of such resolutions correspond to the following four types of protein structures: primary, secondary, tertiary, and quaternary structures. 
The primary structure of a protein is defined as the sequence of the constituent residues of this protein.{}
The covalent bonds in a protein fully determine the primary structure of this protein. 
The secondary structure of a protein is defined as the position of every residue of this protein relative to the residues that are sequentially near this residue.
The non-covalent bonds in a protein fully determine the secondary structure of this protein.{}
Alpha helices and beta strands are the most common protein secondary structures.
Random coil is defined as the lack of any secondary structure. 
The tertiary structure of a protein is defined as the three-dimensional shape of this protein.
The quaternary structure of a protein is defined as the three-dimensional shape of this protein when interacting with other macromolecules. 

Determination of the primary structure of a protein is very easy. 
The state-of-the-art predictors of protein secondary structure achieve an accuracy of approximately 90\% \cite{6217208}.
Moreover, a lot of experimental methods such as nuclear magnetic resonance (NMR) can determine protein secondary structure.
Thus, determination of the secondary structure of a protein is easy.
Experimental methods for determining the tertiary structure of a protein are labor-intensive and time-consuming,
	and computational methods for determining the tertiary structure of a protein suffer from exponential runtime complexity and are not reliable.
Thus, determination of the tertiary structure of a protein is hard.
It is very hard to determine the quaternary structure of a protein. 
Nowadays, determination of the tertiary structure of a protein is still a major challenge. 
In this thesis, ``protein structure'' refers to only ``tertiary protein structure'' unless explicitly stated otherwise. 

Protein structure is not static. 
The structure of a protein depends on the physiological environment that surrounds this protein. 
For example, in an acidic solution at pH\(\approx\)2, a protein usually does not assume any shape at all. 
Thus, every residue of this protein is exposed to this acidic solution, and this protein is referred to as denatured.
Denaturation is defined as the process by which a macromolecule loses its quaternary structure, tertiary structure, and then secondary structure.   
We can view the structure of a protein as a point on a high-dimensional energy landscape. 
On this landscape, the altitude of a coordinate represents the energy of a conformation. 
Every protein tends to adopt a low-energy conformation.
This tendency is consistent with the second law of thermodynamics because high entropy is associated with low-energy conformation. 
A change in the coordinate of a point on the energy landscape corresponds to a change in the conformation of a protein. 
A path from one coordinate to another coordinate on the energy landscape corresponds to a transition from one conformation to another conformation. 
When a new protein is just synthesized, this new protein usually does not have its intended structure and thus is usually not functional yet.{} 
Then, the structure of this new protein changes until this structure stabilizes at a local minimum on the energy landscape.{} 
Then, this protein is able to perform its intended function due to its stable structure and is thus functional.
The process by which a protein becomes functional by assuming its intended structure is referred to as protein folding.
This intended structure is referred to as its native structure.

A few methods have been developed for studying protein structures. 
Unfortunately, all these methods have some weaknesses.
For example, X-ray crystallography is neither effective for membrane proteins nor effective for studying protein-folding dynamics;
	NMR is labor-intensive, time-consuming, and not effective for studying protein-folding dynamics.  

The \glsfirst{SASA}, also known as accessible surface area (ASA), of a biomolecule is the part of the surface area of this biomolecule such that this part can be accessed by the solvent in which this biomolecule is dissolved.{}
In protein-related scientific fields, this biomolecule usually refers to a protein of interest and this solvent usually refers to water.
Change in the	\gls{SASA} of a protein as a function of the time since this protein started to fold can reveal the following trend:
	this protein's surface that is exposed to water as a function of this time.
This trend can then partially reveal the folding dynamics of this protein.
		
Any amino-acid residue of any protein can be covalently modified by the solution that contains this protein.	
Let us assume that a treatment does not change the composition of the solution for a substantial period of time.
If this solution covalently modifies this residue more heavily after this treatment, then this residue is more solvent-accessible after this treatment, and vice versa.
Thus, change in the extent of this covalent modification on this residue is positively correlated with change in the solvent-accessibility of this residue.
Thus, change in the extent of this covalent modification on every residue of a protein of interest can reveal change in \gls{SASA} of this protein. 
		
\Gls{RP-MS} can estimate the \gls{SASA} of a protein.
An \gls{RP-MS} experiment usually proceeds as follows.
First, a protein of interest is tuned to be at a given stage of protein folding.
Next, a source of energy such as ultraviolet light generates short-lived free radicals such as \gls{OH-rad}.{}
Then, these free radicals cause covalent modifications, which mainly consist of \gls{ox1}, to the solvent-accessible surface of this protein of interest.
Afterwards, a protease such as trypsin cleaves this protein of interest into shorter peptides which could have been modified by these free radicals.
Finally, \gls{LC} elutes and thus separates the mixture of these shorter peptides.{}
While \gls{LC} is eluting these shorter peptides, \gls{MS} identifies these peptides and quantitates the extent of oxidation on each of theses peptides.
A protein of interest goes through different stages in the folding process of this protein.
These different stages result in different extents of covalent modification on the residues of this protein of interest, respectively.
Thus, changes in the \gls{SASA} of this protein of interest across these different stages can be inferred,
Thus, \gls{RP-MS} can be used for studying protein-folding dynamics.
Moreover, \gls{RP-MS} is fast, cost-efficient, applicable to any protein, and able to detect a rapid change in protein structure.
Unfortunately, a protease usually only cleaves a protein at a few backbones of this protein. % or equivalently cleavage sites
Thus, a protease usually cleaves a protein into only a few long peptides.
Thus, each of these long peptides is composed of several residues.
Thus, the spatial resolution of \gls{RP-MS} is limited to the peptide level that is determined by this protease.
