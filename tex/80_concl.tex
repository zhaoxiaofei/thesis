%--------1---------2---------3---------4---------5---------6---------7---------8---------9---------1---------2---------3---------4---------5---------6
%23456789 123456789 123456789 123456789 123456789 123456789 123456789 123456789 123456789 123456789 123456789 123456789 123456789 123456789 123456789

\glsunsetall
\chapter{Conclusion}
\label{chap:concl}
\glsresetall

% -> Why did I do it?
The spatial resolution of the quantitation of oxidation by \gls{RP-MS} is low. 
The low spatial resolution of such quantitations result in the low spatial resolution of the \gls{SASA} derived from such quantitations.
The low spatial resolution of this \gls{SASA} results in the low spatial resolution at which protein folding is studied.
% -> What did I do?
We showed that targeted \gls{LC-MS/MS} can improve the spatial resolution of such quantitations of oxidation.
Moreover, we designed an algorithm that automates such quantitations of oxidation at this improved spatial resolution.
% -> How did I do it?
\Gls{MS/MS} can fragment a \gls{mono-oxidized} peptide into the suffixes of this peptide.
Thus, one such suffix is oxidized if and only if the oxidation site on this peptide is on this suffix.
Thus, one such suffix of length \(i\) is oxidized if and only if this oxidation site is in the last \(i\) amino-acid residues of this peptide.
Let \(\phi_i\) be the relative frequency that one such suffix of length \(i\) is oxidized.
Without loss of generality, let \(i>j\).
Then, \(\phi_i - \phi_{j}\) denotes the relative frequency that the oxidation site in a given \gls{mono-oxidized} peptide is between the \(i\)\textsuperscript{th}-last and \((j+1)\)\textsuperscript{th}-last amino-acid residues of this peptide.
Thus, \(\phi_i - \phi_{j}\) can be used for quantitating oxidation at subpeptide level,
	and \(\phi_i - \phi_{i-1}\) can be used for quantitating oxidation at residue level.
% -> what proves that my result is good and correct?
We evaluated our algorithm on an \gls{MS/MS} dataset, most of which is produced by six runs of targeted \gls{MS/MS}.
Our algorithm quantitated oxidation near residue level.
The extents of oxidation computed by our algorithm agree with the corresponding theoretical extents of oxidation.
Thus, our algorithm is sufficiently correct.
% -> limitations
The throughput of targeted \gls{LC-MS/MS} is low.
Also, the fragmentation chemistry in \gls{MS2} can result in a bias in the oxidation quantitated by our algorithm.
However, our algorithm is still sufficiently useful.	

% -> Why did I do it?
However, random errors exist in such quantitation of oxidation. 
Worse yet, only multiple repeated runs can empirically estimate such random errors, but we have only one run of targeted \gls{LC-MS/MS} per peptide.
% -> What did I do?
%Thus, we proposed an empirical formula.
%Our empirical formula estimates such random errors in \(\phi_i\) even when the number of such runs is insufficient.
%Moreover, our empirical formula is applicable to a \gls{AUCXIC} fraction which is a generalized version of \(\phi_i\).
% -> How did I do it?
To estimate such random errors using insufficient experiment data, we made some assumptions partially supported by evidence in the literature.
Then, from these assumptions, we mathematically deduced an empirical formula.
Our empirical formula estimates the random error in the \gls{AUCXIC} fraction that is calculated from only one run of \gls{LC-MS/MS}.
A \gls{AUCXIC} fraction represents, in a sample of interest, the quantity of a type of molecules relative to another type of molecules.
\gls{AUCXIC} fraction is a generalized version of \(\phi_i\).
% -> what proves that my result is good and correct?
Three nearly repeated runs of \gls{LC-MS/MS} confirmed that our empirical formula is sufficiently correct.
% -> limitations
These three runs are all performed by only a \gls{QTOF} mass spectrometer that analyzed only a non-complex sample.
Multiple runs intrinsically provide more information than one run.
Thus, one run generally cannot replace repeated runs.
For example, multiple repeated runs respectively result in multiple estimates of the same expected value of a \gls{AUCXIC} fraction.
Then, the average of these multiple estimates has less random error than any one of these multiple estimates.
However, our empirical formula is still sufficiently useful.

The throughput of targeted \gls{MS/MS} is lower than the throughput of \gls{MSE} by orders of magnitude.
Thus, we hypothesized that \gls{MSE} can also improve the spatial resolution of \gls{RP-MS}.
Unfortunately, an \gls{MSE} dataset shows that our hypothesis is likely to be wrong.
Moreover, an additional \gls{MSE} dataset shows that our hypothesis is very likely to be wrong.
		
\Gls{MSE} does not seem to be able to achieve the purpose of improving the spatial resolution of \gls{RP-MS}.	
Thus, in the future, we will try some alternative approaches for this purpose.
Ideally, these alternative approaches should be neither labor-intensive nor time-consuming.
For example, the following experimental methods all outperform \gls{MSE} in protein identification:
	ion mobility spectrometry (IMS) assisted \gls{MSE} (HD-\gls{MSE}) \cite{distler2013drift},
	ultra-definition \gls{MSE} (UD-\gls{MSE}) \cite{distler2013drift},
	and multiplexed \gls{MS/MS} \cite{egertson2013multiplexed}.
Thus, these methods can be the basis of these alternative approaches.

In the future, we will also test our empirical formula on additional datasets.
For example, one such additional dataset can be produced by a mass spectrometer of another type,
	and this mass spectrometer can analyze a complex sample to produce this dataset.	
In isobaric tags for relative and absolute quantitation (iTRAQ) experiments,	
	the ratio of iTRAQ reporter ions is also a \gls{AUCXIC} fraction.
Thus, our empirical formula has the potential to estimate the random error in iTRAQ when fewer-than-expected runs of \gls{LC-MS/MS} are performed.
