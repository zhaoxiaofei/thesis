%--------1---------2---------3---------4---------5---------6---------7---------8---------9---------1---------2---------3---------4---------5---------6
%23456789 123456789 123456789 123456789 123456789 123456789 123456789 123456789 123456789 123456789 123456789 123456789 123456789 123456789 123456789

% The \appendix statement indicates the beginning of the appendices.
\appendix

% Add a title page before the appendices and a line in the Table of Contents
\chapter*{APPENDICES}
\addcontentsline{toc}{chapter}{APPENDICES}

\section*{Our custom peak-detection algorithm}

The intensity of a candidate peak is assumed to be proportional to the probability that this candidate peak is a true peak.
And the respective generations of different peaks are assumed to be pairwise independent.
Thus, the product of the respective intensities of different candidate peaks is assumed to be proportional to the probability that these candidate peaks are all true peaks.
These candidate peaks can be respectively in different mass spectra but all have the same \gls{m/z}.		
The continuous range of applicable \gls{m/z} is partitioned into connected and pairwise non-overlapping subranges of \gls{m/z}. 
Each of these subranges spans 0.01 \gls{m/z}.
Let \(p\) be probability that a first subrange contains at least one true peak.
Let \(p'\) be the probability that another subrange near this first subrange contains at least one true peak.
Then, \(p\) relative to \(p'\) is the likelihood that this first subrange contains at least one significant true peak.
If this significant true peak does not overlap with any other peak that has already been picked, then this significant true peak is picked.
		
\glsunset{RT}
\begin{algorithm}
\def\MZ{\tfrac{M}{Z}}
\def\mz{\tfrac{m}{z}}
\def\inten{\text{Intensity}}
\def\moz{\text{mz}}	
\def\loginten{\text{sumLnInts}}
\def\loglike{\text{lnLike}}
\def\SA{{\mathit{1}}}
\def\SB{{\mathit{2}}}
\def\SX{{\mathit{k}}}
\caption{detect-peaks(\(r_{\selectAnyFirst}\), \(r_{\selectAnySecond}\)) %, $\mz$, $t$
\label{alg:NM:methods:peakdetection}
}
\begin{algorithmic}[1]
\Require{
	%$s$ is a sequence of \gls{MS2} spectra, $s'$ is another sequence of \gls{MS2} spectra;
	%\(\{s, s'\}\) is \gls{iid};
	%$s$ and $s'$ are produced from runs of \gls{LC-MS/MS} that take as input identical samples;
	\(r_{\selectAnyFirst} \) is a sequence of \gls{MS2} spectra produced by a first  run of \gls{LC-MS/MS}, 
	\(r_{\selectAnySecond}\) is a sequence of \gls{MS2} spectra produced by a second run of \gls{LC-MS/MS};
	\(\big(r_{\selectAnyFirst}, r_{\selectAnySecond}\big)\) should be \gls{iid},
		or equivalently the run that produced \(r_{\selectAnyFirst}\) and the run that produced \(r_{\selectAnySecond}\) should be {repeated}.		
%	\(\forall s_{\bullet} \in s, s_{\bullet}' \in s' \colon 
%		\text{premz}(s_{\bullet}) \approx \text{premz}(s_{\bullet}') \land
%	 	\glstext{RT}(s_{\bullet}) \approx \glstext{RT}(s_{\bullet}')
%	%\text{ are chemically identical to } \text{precursors}(s_{\bullet}') 
%	\);
%	\(\text{premz}(s_{\bullet}^{_{\circ}})\) is the \gls{m/z} of precursor ions selected for \gls{MS2}, 
%		where such \gls{MS2} generates the \gls{MS2} spectrum \(s_{\bullet}^{_{\circ}}\);
%		a mass spectrum is a sequence of (\gls{m/z}, intensity).
	%where \(\text{precursor}(s_{\bullet}^{_{\circ}})\) is the set of all precursor-ion types that generate the \gls{MS2} spectrum \(s_{\bullet}^{_{\circ}}\).
	 %of precursors generated both $S_\SA'$ and $S_\SB'$.		
}
\Ensure{the set of all \gls{m/z},
		where each of these \gls{m/z} is the \gls{m/z} of a chemical species of product ions that is detected in both 
			\(r_{\selectAnyFirst}\) and \(r_{\selectAnySecond}\).}
\newline\textbf{We manually validated by visual inspection the output of this algorithm.}
\State find a subsequence \(r_{\selectAnyFirst}'\) of \(r_{\selectAnyFirst}\) 
		and a subsequence \(r_{\selectAnySecond}'\) of \(r_{\selectAnySecond}\)
		such that
		\begin{enumerate}
		\item
		the \gls{RT} of any spectrum \(s_{\selectAnyFirst}\)  in \(r_{\selectAnyFirst}'\) \(\approx\) 
		the \gls{RT} of any spectrum \(s_{\selectAnySecond}\) in \(r_{\selectAnySecond}'\) and
		\item
		the precursor \gls{m/z} of any \(s_{\selectAnyFirst}\)  in \(r_{\selectAnyFirst}'\) \(\approx\) 
		the precursor \gls{m/z} of any \(s_{\selectAnySecond}\) in \(r_{\selectAnySecond}'\).
		\end{enumerate}
\State \(\displaystyle \varDelta \isdefinedas \{\frac{-50}{100}, \frac{-49}{100}, ..., \frac{49}{100}, \frac{50}{100}\}\)
		\Comment{Discretized values of \gls{m/z}}
\State \(\displaystyle\loginten(\mz, r')
		\isdefinedas\sum_{s \in r'}\left(\ln\left(1+\sum_{p \in s}\left(\inten(p)\cdot\mathbbm{1}[\mz < \gls{m/z} \text{ of } p \le \mz + \frac{1}{100}]\right)\right)\right)\)
			\newline\Comment{sum of logarithm of peak intensity in every spectrum with add-one Laplace smoothing,
			where \(p\) means peak, \(s\) means spectrum, and where \(r'\) means sequence of spectra}
\State \(\displaystyle\loglike(\mz, r') 
		\isdefinedas \loginten(\mz, r') - \frac{1}{101}\cdot\sum_{\delta \in \varDelta}	 \big(\loginten(\mz+\delta, r')\big)\)
			\newline\Comment{log-likelihood at an \gls{m/z} relative to the background log-likelihood near this \gls{m/z}}
%\begin{align}
%\displaystyle\loginten(\mz', S'')
%		&\isdefinedas\sum_{s'' \in S''}\bigg(\ln\Big(1+\sum_{p \in s''}\big(\inten(p)\cdot\mathbbm{1}[\mz' < \moz(p) \le \mz' + \frac{1}{100}]\big)\Big)\bigg) \\
%\displaystyle\loglike(\mz', S'') 
%		&\isdefinedas \loginten(\mz', S'') - \frac{1}{101}\cdot\sum_{\delta \in \varDelta}	 \big(\loginten(\mz'+\delta, S'')\big)
%\\ \text{where~~} \varDelta& \isdefinedas \{\frac{-50}{100}, \frac{-49}{100}, ..., \frac{49}{100}, \frac{50}{100}\}
%\end{align}		
\State \(\displaystyle \MZ \isdefinedas \{\frac{1}{100}, \frac{2}{100}, ..., \frac{199999}{100}, \frac{200000}{100}\}\),
		\(\displaystyle \MZ' \isdefinedas \emptyset\)
		\Comment{Discretized values of \gls{m/z}}
\While {\(\displaystyle\max_{\mz\in\MZ}\loglike(\mz, r_{\selectAnyFirst}') > \frac{\ln(200000)}{10} \land |\MZ'|<1000\)}
	\State \(\displaystyle\mz\isdefinedas \argmax_{\mz\in\MZ}\loglike(\mz, r_{\selectAnyFirst}')\)
	\State \(\displaystyle\MZ\isdefinedas \MZ \setminus \left(\bigcup_{\delta\in\varDelta} \left(\{\mz+\delta\}\right) \right)\)
			\Comment{eliminate peaks that are adjacent in \gls{m/z}} 
	\If{\(\displaystyle\loglike(\mz, r_{\selectAnySecond}')> \frac{\ln(200000)}{10} \land (\forall \mz'\in\MZ': {-3} < \mz'-\mz < 3)\)} 
		\\\Comment{select intensity with high relative log-likelihood and naively avoid isotope}
		\State \(\displaystyle\MZ'\isdefinedas \MZ' \cup \{\mz\}\)
	\EndIf
\EndWhile
\State \Return \(\MZ'\)
\end{algorithmic}
\end{algorithm}

\section*{Additional justification for using our empirical formula}

\Cref{tab:NM:dataset:pep-seq} summarizes the dataset used for testing our empirical formula.
\Cref{tab:oxlvl:6-tryptic-peptides} summarizes the dataset to which our empirical formula is applied.
The comparison between these two datasets reveals the following discrepancy. 
The \gls{RT} ranges in \cref{tab:oxlvl:6-tryptic-peptides} are typically much larger than the \gls{RT} ranges in \cref{tab:NM:dataset:pep-seq}.
Thus, the following potential problem arises.
Our empirical formula is perhaps not applicable to the dataset summarized in \cref{tab:oxlvl:6-tryptic-peptides}, 
	because the errors in a smaller \gls{RT} range cannot be extrapolated to the errors in a larger \gls{RT} range.
	
Fortunately, our empirical formula is not affected by this potential problem. 
The following two paragraphs explain why this potential problem is actually not a problem.

First, each of the peptides in \cref{tab:oxlvl:6-tryptic-peptides} has most of its peak intensities concentrated at a few short \gls{RT} intervals.
The length of each of these \gls{RT} intervals is similar to the length of any \gls{RT} range in \cref{tab:NM:dataset:pep-seq} (data not shown).
Thus, the union of these \gls{RT} intervals is still larger, but not much larger, than a typical \gls{RT} range in \cref{tab:NM:dataset:pep-seq}.
Thus, this problem is alleviated.

Second, we merged the \gls{MS2} spectra of each peptide in \cref{tab:oxlvl:6-tryptic-peptides}. 
The number of these merged \gls{MS2} spectra is approximately the number of the \gls{MS2} spectra of each peptide in \cref{tab:NM:dataset:pep-seq}.
This merging step does not increase any error.
After this merging step, the following is observed.
The multiplicative random error captured by the constant \(\delta\) is still negligible compared with the shot noise captured by \cref{eq:NM:derivation:simplificationresult}.
Also, this multiplicative random error and this shot noise should constitute most of the error in these merged mass spectra.
Thus, the shot noise captured by \cref{eq:NM:derivation:simplificationresult} is sufficient for characterizing the random errors in the dataset summarized in \cref{tab:oxlvl:6-tryptic-peptides}.  	

%% To include a Nomenclature section
%\addcontentsline{toc}{chapter}{\textbf{Nomenclature}}
%\renewcommand{\nomname}{Nomenclature}
% %\printglossary[title=Nomenclature,toctitle=Nomenclature]
\printglossary[type=s-this-symbs]
\printglossary[type=c-math-defns]
\printglossary[type=c-rpms-defns]
\printglossary[type=\acronymtype,title=Acronyms in mass spectrometry]