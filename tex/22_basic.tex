%--------1---------2---------3---------4---------5---------6---------7---------8---------9---------1---------2---------3---------4---------5---------6
%23456789 123456789 123456789 123456789 123456789 123456789 123456789 123456789 123456789 123456789 123456789 123456789 123456789 123456789 123456789

\glsunsetall
\chapter{Fundamentals of \texorpdfstring{\glsfirst{MS}}{mass spectrometry}}
\label{chap:fund2}
\glsresetall

This chapter presents the fundamentals of \gls{MS} and focuses on \gls{MS}-related concepts.
The order in which we present the sections in this chapter is approximately the order in which a typical \gls{RP-MS} experiment is performed.

\Cref{sec:MS:FPOP} presents \gls{FPOP}, an analytical-chemistry technique for oxidizing an investigated protein.
\Cref{sec:MS:proteolysis} presents proteolysis, a molecular-biology method often applied before performing \gls{HPLC}.
\Cref{sec:MS:proteolysis} presents \gls{HPLC}, an analytical-chemistry technique for separating analytes based on some of their chemical properties.
\Cref{sec:MS:MS} presents \gls{MS}, an analytical-chemistry technique for measuring the \gls{m/z} of analytes.
\Cref{sec:MS:MSMS} presents \gls{MS/MS}, a subclass of \gls{MS}.
In essence, \gls{MS/MS} fragments analytes and then measures the \gls{m/z} of each of these fragments.
\Cref{sec:MS:prep} presents the preprocessing of raw mass spectra.
Examples of such preprocessing are centroiding, deconvolution, and deisotoping.
Such processing facilitates the subsequent interpretation of these preprocessed mass spectra.
\Cref{sec:MS:PSM} presents \gls{PSM}, a key concept in the interpretation of the \gls{MS2} spectra that are produced by protein \gls{MS}.
\Cref{sec:fund2:MS-protocols} presents common protocols of protein \gls{MS} and focuses on \gls{RP-MS}.

\section{\texorpdfstring{\Glsfirst{FPOP}}{FPOP}} 
\label{sec:MS:FPOP}

\Gls{FPOP} is an analytical-chemistry technique for covalently labeling a protein.
		%\Gls{OH-rad} is the electrically neutral form of hydroxide ion. 
\Gls{FPOP} uses ultraviolet light to cause the dissociation of hydrogen peroxide (\HydrogenPeroxide{}) and then the formation of \gls{OH-rad}, 
%or by the photolysis of 1-Hydroxy-2(1H)-pyridinethione. 
The action of such ultraviolet light is shown in the following chemical equation. % shows the dissociation of \HydrogenPeroxide{} under ultraviolet light.
\begin{align*}
\HydrogenPeroxide{} + {\lightning} ~ \to ~ \text{HO} {\lightning} \text{OH} ~ \to ~  2 ~ \gls{OH-rad}
& & ({\lightning}\text{ denotes 248 nm ultraviolet light})
\end{align*} % 254 nm ?
The \textit{in vivo} half-life of \gls{OH-rad} is approximately \(10^{-9}\si{\second}\), and \gls{OH-rad} is highly reactive. % wikipedia [4]
Thus, \gls{OH-rad} virtually damages all types of macromolecules such as carbohydrates, nucleic acids, lipids, and amino acids \cite{JPI:JPI1}. 

\begin{figure}
\center
\myframe{\includegraphics[width=\textwidth, page=1,   clip=true, bb=37 109 551 240]{img/gruebele2010analytical.pdf}}
\caption[
	An overview of \gls{FPOP} \cite{gruebele2010analytical}.]{
	An overview of \gls{FPOP} \cite{gruebele2010analytical}.
	A protein experiences four stages while passing through a tube. 
	(a) A denatured protein and hydrogen peroxides (\HydrogenPeroxide{}) flow through a capillary tube. 
	(b) An infrared laser having a wavelength of 1900 nanometers initiates the folding of the denatured protein.
	(c) After a brief delay, an ultraviolet laser having a wavelength of 248 nanometers 
	    		splits \HydrogenPeroxide{} into \gls{OH-rad}.
	    	Some of these \gls{OH-rad} almost instantaneously oxidize some residues of the partially folded protein.
	    	Longer delay causes the protein to be less denatured before \gls{OH-rad} is generated, 
	    		then this less denatured protein is subject to less modification by \gls{OH-rad}.
	    	If a residue of the protein is closer to the solvent that dissolved the protein,
	    		then this residue is more heavily modified by \gls{OH-rad}.
	    	The addition of red dots to the protein 
	    		represents the addition of oxygen atoms to the protein.
	(d) The modified protein exits the capillary, 
	    	a radical scavenger removes the remaining \HydrogenPeroxide{},
	    	and \gls{MS} quantitates the \gls{OH-rad}-mediated modification which is positively correlated with \gls{SASA}.
	    	Before \gls{MS}, 
	     	proteolysis and then \gls{LC} are often performed on the modified protein.      	
	\label{fig:fund2:schematics-FPOP}}
\end{figure}

\Gls{OH-rad} can covalently modify a residue through different mechanisms.
Thus, different mass shifts of the modified residue can be observed.
The time scale of these covalent modifications is usually less than one millisecond.
Thus, there is only a sub-millisecond interval between the time that \gls{OH-rad} enters in contact with a residue 
		and the time that \gls{OH-rad} finishes covalently modifying this residue. 
While \gls{OH-rad} is covalently modifying a protein, \gls{OH-rad} may also affect the folding process of this protein.
However, virtually all protein folding processes take more than one millisecond to initiate.
Thus, the time scale of covalent modification mediated by \gls{OH-rad} is short compared with the time scale of protein folding.
Thus, before that \gls{OH-rad} finishes covalently modifying a protein, the overall folding process of this protein is unlikely to be affected.
In fact, \gls{OH-rad} indeed does not substantially affect the folding process of a protein before it finishes modifying this protein \cite{gau2011advancement}.

Intuitively, if the solvent accessibility of a residue is high, then this residue is more likely to be covalently modified by \gls{OH-rad}.
%Thus, if solvent accessibility is the only explanatory variable which is different in more than one runs of \gls{FPOP}, 
%		then the more heavily a residue is covalently modified by \gls{OH-rad}, the more likely that this residue is more accessible by solvent.{}
Thus, the \gls{SASA} %, or equivalently the accessible surface area (ASA), 
	of a residue of a protein is positively correlated with the extent of \gls{OH-rad}-mediated covalent modification on this residue.
Moreover, the duration of \gls{OH-rad}-mediated covalent modification is short compared with the duration of protein folding. 
Thus, the \gls{OH-rad}-mediated modification to any protein only depends on the structure of this protein at the precise time of this modification,
Thus, after a protein refolds for a given amount of time,
		the extent of the \gls{OH-rad}-mediated modification to each residue of this protein reveals the \gls{SASA} of this protein after this amount of time.
Thus, if the time taken for a protein to refold varies,
	%where this refolding event happens after this protein starts to refold and before that this protein interacts with \gls{OH-rad},
	then the extent of the \gls{OH-rad}-mediated modification to any residue of this protein as a function of this time 
	is positively correlated with the solvent-accessibility of this residue as a function of this time.{}
Thus, we can characterize the change in the \gls{SASA} of a protein as a function of time and thus study protein-folding dynamics.

\begin{table}
\centering{
  \begin{tabular}{ l l l l l l l l }
  \toprule
  \multicolumn{2}{l}{\(\begin{subarray}{l}\displaystyle\text{Free}\\\displaystyle\text{amino acid}\end{subarray}\)} 
  & \(\begin{subarray}{l}\displaystyle\text{k}_{\gls{OH-rad}} \\ (\textsc{M}^{-1}\si{\second}^{-1}) \end{subarray}\) \cite{buxton1988critical}        
  & \multicolumn{5}{l}{\(\begin{subarray}{l}\displaystyle\text{Common mass shifts}
  	\\\displaystyle\text{resulting from modification by \gls{OH-rad} (\si{\dalton}) \cite{xu2007hydroxyl}}\end{subarray}\)} \\ \midrule
  %ref THE ADVANCEMENT OF MASS SPECTROMETRY
  Cys&(C)    & \(3.5 \times 10^{10}\)        & \(-15,9772\) & \(+31.9898\) & \(+47.9847\) & & \\
  Trp&(W)    & \(1.3 \times 10^{10}\)        & \(+3.9949 \) & \(+15.9949\) & \(+31.9898\) & \(+47.9847\) & \\ 
  Tyr&(Y)    & \(1.3 \times 10^{10}\)        & \(+15.9949\) & \(+31.9898\) & \(+47.9847\) & & \\ 
  Met&(M)    & \(8.5 \times 10^9 \)          & \(-32.0085\) & \(+15.9949\) & \(+31.9898\) & & \\ 
  Phe&(F)    & \(6.9 \times 10^9 \)          & \(+15.9949\) & \(+31.9898\) & \(+47.9847\) & & \\ 
  His&(H)    & \(4.8 \times 10^9 \)          & \(-23.0160\) & \(-22.0320\) & \(-10.0320\) & \(+4.9789\) & \(+15.9949\) \\ 
  Arg&(R)    & \(3.5 \times 10^9 \)          & \(-43.0534\) & \(+13.9793\) & \(+15.9949\) & & \\ 
  Ile&(I)    & \(1.8 \times 10^9 \)          & \(+13.9793\) & \(+15.9949\) & & & \\ 
  Leu&(L)    & \(1.7 \times 10^9 \)          & \(+13.9793\) & \(+15.9949\) & & & \\ 
  Val&(V)    & \(8.5 \times 10^8 \)          & \(+13.9793\) & \(+15.9949\) & & & \\ 
  Pro&(P)    & \(6.5 \times 10^8 \)          & \(+13.9793\) & \(+15.9949\) & & & \\ 
  Gln&(Q)    & \(5.4 \times 10^8 \)          & \(+13.9793\) & \(+15.9949\) & & & \\ 
  Thr&(T)    & \(5.1 \times 10^8 \)          & \( -2.0157\) & \(+15.9949\) & & & \\ 
  Lys&(K)    & \(3.5 \times 10^8 \)          & \(+13.9793\) & \(+15.9949\) & & & \\ 
  Ser&(S)    & \(3.2 \times 10^8 \)          & \( -2.0157\) & \(+15.9949\) & & & \\ 
  Glu&(E)    & \(2.3 \times 10^8 \)          & \(-30.0106\) & \(+13.9793\) & \(+15.9949\) & & \\ 
  Ala&(A)    & \(7.7 \times 10^7 \)          & \(+15.9949\) & & & & \\ 
  Asp&(D)    & \(7.5 \times 10^7 \)          & \(-30.0106\) & \(+15.9949\) & & & \\ 
  Asn&(N)    & \(4.9 \times 10^7 \)          & \(+15.9949\) & & & & \\ 
  Gly&(G)    & \(1.7 \times 10^7 \)          & n.d. & & & & \\ \bottomrule
  \end{tabular}}
\caption[
	The initial rates of the second-order reaction of free amino acids with \gls{OH-rad} at pH=7 
			and the common mass shifts produced by such reaction.]{
	The initial rates of the second-order reaction of free amino acids with \gls{OH-rad} at pH=7 
			and the common mass shifts produced by such reaction.
	A free molecule is not covalently bound to any other molecule. 
	If an amino acid residue is part of a given protein in a given solution, 
		then the \(\text{k}_{\gls{OH-rad}}\) of this residue depends on 
			the position of this residue with respect to this given protein and on the properties of this given solution.
	\label{tab:AA-OH-reaction-rate}}

\begin{figure}[H]
\center
\myframe{\includegraphics[width=\textwidth, page=4,   clip=true, bb=143 397 472 441]{img/lee2006perr.pdf}}
\caption[
	A mechanism of histidine oxidation \cite{lee2006perr}.]{
	A mechanism of histidine oxidation \cite{lee2006perr}.
	\label{OH_Leu_example}}
\end{figure}
\end{table}	

Different residues have different mechanisms of reacting with \gls{OH-rad}. 
Thus, different residues have hugely different reaction rates with \gls{OH-rad}.
For example,	the second order reaction rate of \gls{OH-rad} with cysteine, the most reactive one, 
 is approximately 2000 times higher than such rate with glycine, the least reactive one \cite{buxton1988critical}.	
However, \gls{OH-rad} usually oxidizes a residue, and oxidation of a residue by \gls{OH-rad} often adds one oxygen atom to this residue.
Thus, \gls{ox1} is the principal \gls{OH-rad}-mediated covalent modification to all residues.
For example, all residues can have a mass shift of +15.9949 or +31.9898 \ensuremath{\si{\dalton}} after reacting with \gls{OH-rad} except glycine, 
	which simply does not react with \gls{OH-rad}.
The mass shifts of +15.9949 and +31.9898 correspond to the addition of one and two oxygen atoms respectively \cite{xu2007hydroxyl}.
\Cref{tab:AA-OH-reaction-rate} shows that reactions between different residues and \gls{OH-rad} happen at different speeds. 
However, these reactions result in similar mass shifts to these different residues.
We use \( +15.9949\), \(+15.99\), and \(+16\) interchangeably to denote the mass shift caused by the addition of one oxygen atom.

\section{Enzymatic proteolysis} 
\label{sec:MS:proteolysis}

\begin{figure}
\center
\begin{tabular}{l l}
{Before any cleavage:}          & \texttt{K-A-F-A-R-W-A-R-P-K-P-R-E-Y-M-Q-F-P-W-P-Y-P}\\
{After trypsin cleavage:}       & \texttt{K|A-F-A-R|W-A-R-P-K-P-R|E-Y-M-Q-F-P-W-P-Y-P}\\
{After} {chymotrypsin cleavage:}& \texttt{K|A-F|A-R|W|A-R-P-K-P-R|E-Y|M-Q-F-P-W-P-Y-P}
\end{tabular}
\caption[An example of proteolysis]
{An example of proteolysis by trypsin and then by high-specificity chymotrypsin.
Unlike most proteases, chymotrypsin does not cleave any polypeptide until trypsin activates this chymotrypsin.
Trypsin cleaves after any of \{\texttt{K}, \texttt{R}\} that is not before \texttt{P}.
High-specificity chymotrypsin cleaves after any of \texttt{\{F, Y, W\}} that is not before \texttt{P}.
\label{fig:MS:proteolysisExample}}
\end{figure}

A protease (also called peptidase, proteinase, or proteolytic enzyme) is an enzyme that can perform proteolysis.
Proteolysis is the cleavage a polypeptide into shorter peptides or amino acids.
A specific protease only cleaves at specific peptide bonds in the backbone of a polypeptide.
Thus, cleavage of a given polypeptide by a given specific protease produces a predictable set of peptides. 

If a carbonyl-carbon is part of lysine (\texttt{K}) or arginine (\texttt{R}) and a nitrogen is not part of proline (\texttt{P}),
	then the protease trypsin cleaves at the peptide bond between this carbonyl-carbon and this nitrogen, and vice versa. 
Equivalently, the specificity rule of trypsin cleavage is referred to as follows: after \texttt{K} or \texttt{R} and not before \texttt{P}. 
Trypsin molecules can cleave each other because trypsin is also a type of protein. 
Thus, trypsin molecules are stored at below \(-20\si{\celsius}\) to prevent them from cleaving each other.
Trypsin is the most commonly used protease for \gls{MS}.

Different specific proteases are subject to different specificity rules. 
For example, 
	LysN cleaves before \texttt{K}, 
	LysC cleaves after  \texttt{K}, 	
	GluC cleaves after  \texttt{E}, 
	AspN cleaves before \texttt{D}, 
	high-specificity chymotrypsin cleaves after any of \{\texttt{F}, \texttt{Y}, \texttt{W}\} and not before \texttt{P},
	and low-specificity chymotrypsin cleaves after any of \{\texttt{F}, \texttt{Y}, \texttt{W}, \texttt{M}, \texttt{L}\} and not before \texttt{P}. 
Different proteases usually cleave the same peptide independently of each other (\cref{fig:MS:proteolysisExample}).
%if different proteases cleave the same polypeptide, 
%then the cleavage of this polypeptide by one of these proteases is usually not substantially affected by other proteases. 

\section{\texorpdfstring{\Glsfirst{HPLC}}{HPLC}} 
\label{subsection_UPLC}

\Glsfirst{LC} is an analytical-chemistry technique.
\Gls{LC} separates a mixture into the components constituting this mixture based on the chemical properties of these components.
In analytical chemistry, components, analytes, constituents, and substances are all equivalent in meaning. %?
To separate a mixture into the components constituting this mixture, \gls{LC} uses a column that elutes different components at respectively different speeds. 

Mobile phase is defined as the solution that is gradually eluted by the column.
Stationary phase is defined as the sorbent of the column which retains components in the solution. 
The \gls{RT} of a component is defined as the elapsed time during which this component is retained by the \gls{LC} column,
	or equivalently the amount of time taken for this component to go through the \gls{LC} column.
The \gls{RT} of multiple molecules may also refer to the shortest time interval that virtually includes the \gls{RT} of all these molecules. 
The \gls{RT} of a component depends on the interaction between this component and the stationary phase.
This interaction depends on the chemistry of the stationary phase, the chemical properties of this component, and on the composition of the mobile phase. 
Clearly, the more a column retains a component, the higher the \gls{RT} of this component will be in this column, and vice versa. 


\Glsfirst{HPLC} is a subclass of \gls{LC}.
\Gls{HPLC} is characterized by the high pressure applied to the mobile phase. 
This high pressure, which is usually between 50 and 350 bars, reduces the \gls{RT} of all components. 
Thus, compared with traditional \gls{LC}, \gls{HPLC} is characterized by higher resolving power and requires less time per run.
\gls{LC} resolving-power is defined as the ability to distinguish two components with slightly different \glspl{RT}.
Thus, \gls{HPLC} has gradually replaced traditional \gls{LC}. 
%In \gls{HPLC}, the stationary phase is made of granular solid particles of 2 to 50 micrometers in size. 
\Cref{LC_overview} shows some key characteristics of \gls{RPHPLC}, a subclass of \gls{HPLC}.
\begin{figure}
\includegraphics[width=\textwidth]{imgbin/RPLC.pdf}
\caption[
	A schematic of \acrshort{RPHPLC}]{
	A schematic of \acrshort{RPHPLC} featuring a hypothetical \acrshort{RPHPLC} experiment.
	In this experiment, the \acrshort{RT} of hydrophobic molecules and the \acrshort{RT} of very hydrophobic molecules overlap.
	Thus, this experiment cannot separate these two types of molecules.
	\label{LC_overview}	
}
\end{figure}
%Different types of \gls{LC} columns result in different types of \gls{LC} or of \gls{HPLC}. 
%In size-exclusion chromatography, the column preferentially retains small components that have low molecular weight.
%In cation-exchange chromatography, the column preferentially retains positively charged ions.{}
%In  anion-exchange chromatography, the column preferentially retains negatively charged ions.
%In bioaffinity chromatography, the column is made of biologically active substances that have a tendency to attract one or more specific components.
%Thus, the column preferentially retains components that have high affinity with it. 
%In partition chromatography, the column preferentially retains polar components.
%In normal-phase chromatography, the column also preferentially retains polar components,
%	and the elution of a component depends on the steric hindrance of this component in the column. 
In \gls{RPHPLC}, the column preferentially retains hydrophobic components.
Thus, in \gls{RPHPLC}, the \gls{RT} of a hydrophobic component should be higher than the \gls{RT} of a hydrophilic component.
Moreover, exposure of the hydrophobic regions of a component depends on the size and shape of this component.
Thus, in \gls{RPHPLC}, the size and shape of a component affect the \gls{RT} of this component.
Frequently, \gls{RPHPLC} and \gls{MS} are used together.
%\Cref{7_hplc} summarizes different types of \gls{LC} or \gls{HPLC}.
%\glsreset{UPLC}
%\Gls{UPLC} is a subtype of \gls{RPHPLC}, 
%		where \gls{UPLC} is characterized by extremely high pressure applied to the \gls{RPHPLC} column;
%	\gls{UPLC} is also a trademark of Waters Corporation.

The following two types of elution exist in \gls{LC}: isocratic elution and gradient elution. %GRAMMAR: (A) elution or (B) elutions
The composition of the mobile phase is relatively constant during isocratic elution and changing during gradient elution.	
Usually, the variation in the \gls{RT} of a component in isocratic elution is lower than the variation in \gls{RT} of this component in gradient elution. 
Thus, the quantity of an eluted component as a function of \gls{RT} forms a sharper peak in gradient elution compared with isocratic elution.
%Thus, gradient elution is more frequently used than isocratic elution for \gls{RPHPLC}.
In gradient elution, the mobile phase consists of mostly water at the beginning.
As elution progresses, an organic solvent miscible with water is gradually added to the mobile phase. 
In the end, the mobile phase consists of mostly this organic solvent. 
Some commonly used organic solvents for gradient elution are acetonitrile, methanol, and tetrahydrofuran.

%Theoretical physical models can predict the \gls{RT} of a component given the \gls{HPLC} experimental parameters
%		\cite{tarasova2006biolccc};
%	however, the \gls{RT} of a component in a mixture heavily depends on the \gls{HPLC} instrument and on the experimental condition.
%	thus, most \gls{RT} predictors predict the \gls{RT} of a component by using the \gls{RT} of at least one other component,
%		where the same \gls{HPLC} instrument elutes all components under the same experimental condition.
%Due to variation in the composition of the mobile phase, 
%		\gls{RT} in gradient elution is harder to predict than \gls{RT} in isocratic elution; 
%	still, back-calculation, which is currently the state-of-the-arts \gls{RT} predictor, 
%		achieves a \(R^2\) of 0.99996 for predicted \gls{RT} as a function of measured \gls{RT}
%		and a root mean square error of \(\pm1\) second out of \(20\) minutes of total elution time \cite{boswell2011easy}.
  
\section{\texorpdfstring{\Glsfirst{MS}}{MS}}
\label{sec:MS:MS}

\Gls{MS} is an analytical-chemistry technique based on the use of a mass spectrometer.
A mass spectrometer takes as input some analytes and outputs the mass spectra of these analytes.
%An analyte, or equivalently a component, is defined as a substance or chemical constituent that is investigated in an analytical procedure.
A mass spectrum is a continuum of signal intensity as a function of \gls{m/z}, where \gls{m/z} denotes mass-to-charge ratio. 

Mass spectra of analytes can reveal some properties of these analytes. 
Examples of these properties are chemical formula and structural formula. %GRAMMAR: chemical and structural formulas?
Moreover, mass spectra can distinguish between isotopes of a chemical element because isotopes have different masses. 

The history of \Gls{MS} is relatively long. 
At the beginning of the 20\textsuperscript{th} century, \gls{MS} has already been used for separating isotopes.
However, the development of protein \gls{MS}, which is the \gls{MS} for studying proteins, only started at the end of the 20\textsuperscript{th} century. 
Large biomolecules such as proteins tend to fragment into small molecules after being ionized.
Thus, ionization tends to destroy the structural formula of a protein.
Thus, protein \gls{MS} has been a major challenge.
In \citeyear{yamashita1984electrospray}, \citet{yamashita1984electrospray} developed the \gls{ESI} method.
		%the soft desorption ionization methods,
\Gls{ESI} can ionize large biomolecules such as proteins without breaking these biomolecules. 
In \citeyear{tanaka1988protein}, \citet{tanaka1988protein} developed the soft-laser-desorption method.
Soft laser desorption can also ionize large biomolecules without breaking these biomolecules.
%In 2002, the Nobel Prize in Chemistry were awarded to John Bennett Fenn
%	for his work on the ionization of protein \cite{fenn2003electrospray}. 

\begin{figure}
\myframe{\includegraphics[height=\textwidth, page=76, clip=true, bb=143 129 280 562, angle=-90]
	{img/eidhammer2008computational.pdf}}
\caption[
	A schematic of a typical mass spectrometer \cite{eidhammer2008computational}.]{
	A schematic of a typical mass spectrometer \cite{eidhammer2008computational}.
	\label{fig:fund2:ms1-schematics}}
\end{figure} 
A mass spectrometer mainly consists of the following three components: an ion source, a mass analyzer, and a mass detector. 
A typical run of \gls{MS} consists of a sequence of scans.
Each of these scans proceeds as follows.
First, the ion source ionizes a set of analytes coming from the inlet of the mass spectrometer so that these analytes form ions.{}
Next, an extraction system brings these ions from the ion source to the mass analyzer.{} 
Then, the mass analyzer separates these ions according to the \gls{m/z} of these ions.
Afterwards, the mass analyzer sends these separated ions to the mass detector.
Finally, the mass detector measures the quantity of ions at each specific \gls{m/z} to produce a mass spectrum.
%Ions only exist in vacuum because ions are highly reactive.
\Cref{fig:fund2:ms1-schematics} shows a schematic of a typical mass spectrometer.
 
\Cref{subsec:fund2:MS:ionsrc} presents some types of ion sources.
\Cref{subsec:MS:MS:analyzer} presents some types of mass analyzers. 
\Cref{subsec:fund2:MS:detector} presents some types of mass detectors.

\subsection{Ion source}
\label{subsec:fund2:MS:ionsrc}

Ionization can be either hard or soft.
Hard ionization usually fragments analytes, and soft ionization usually does not fragment analytes. 
For example, electron-impact ionization (EI), also known as electron ionization, is a hard ionization technique. 
Some popular soft ionization techniques are 
		fast atom bombardment (FAB), 
		chemical ionization (CI),
		matrix-assisted laser desorption/ionization (\rlink{MALDI}),
		and electrospray ionization (\gls{ESI}).
Among these techniques, only \rlink{MALDI} and \gls{ESI} can ionize large biomolecules without fragmenting most of these biomolecules. 
\Gls{ESI} is currently the most popular ionization technique. 

%The working mechanism of \rlink{MALDI} is the following.
%\begin{enumerate}[nolistsep]
%\item Sample molecules are introduced into a solid crystallized matrix.
%\item This matrix absorbs an ultraviolet laser fired to this matrix. 
%\item This ultraviolet laser ablates the upper layer of this matrix.
%	This ablated layer generates an ionizing plume. % that consists of a mixture of both this upper layer and the sample molecules. 
%\item In this ionizing plume, the sample molecules are ionized. 
%	Most of the sample molecules become singly charged cations.{} 
%	Some of the sample molecules can become singly charged anions.{}
%	Some of the sample molecules can become multiply charged cations.
%\end{enumerate}

\def \M {\textnormal{M}}
\def \H {\textnormal{H}}

The working mechanism of \gls{ESI} is the following.
\begin{enumerate}[nolistsep]
\item By mixing water and volatile compounds, a solvent is prepared. 
	By mixing this solvent with sample molecules, a solution is prepared.
%	A solution is prepared by mixing sample molecules with a solvent. 
%	This solvent consists of water and volatile compounds. %such as methanol and acetonitrile. 
\item This solution is dispersed by an electrospray into aerosol.
\item This aerosol is subject to a strong electric field to produce charged droplets.
\item The solvent in these charged droplets evaporates. 
	During this evaporation, each of these droplets can undergo the subsequent Coulomb-fission cycle.
	\begin{enumerate}[nolistsep,label={\arabic*.}]
	\item Due to this evaporation, one such droplet continuously decreases in size. 
		However, the charge on this droplet remains constant.
	\item The electrostatic repulsion of the same charge on this droplet becomes too high compared with the surface tension that holds this droplet together. 
	\item This droplet explodes and then becomes multiple smaller droplets. 
	%Such an explosion is referred to as Coulomb fission.
	\item Each of these smaller droplets can recursively undergo another such Coulomb-fission cycle.
	\end{enumerate}
\item 
	The solvent is almost completely evaporated.
	Each of the sample molecules might have one or more charges. % that can be either positive or negative.
	%one positive charge, multiple positive charges, one negative charge, multiple negative charges.
\end{enumerate}
\Gls{ESI} has several advantages. 
For example, \gls{ESI} can ionize a protein without denaturing this protein, can analyze a dilute solution, and can ionize analytes in any polar solvent. 
Most importantly, \gls{ESI} can generate multiply charged ions. 
Thus, the \gls{m/z} of a molecule with high molecular weight can still be within the \gls{m/z} detection range of a typical mass spectrometer.
Thus, \gls{ESI} is currently the most commonly used ionization technique. 

\subsection{Mass analyzer} 
\label{subsec:MS:MS:analyzer}

Every mass analyzer separates ions according to the \glsfirst{m/z} of each of these ions. 
\Gls{m/z} \glsdesc{m/z} %GRAMMAR: spacing looks off
Every mass analyzer is characterized by \gls{m/z} range, peak shape, mass resolution (resolution), and mass accuracy (accuracy).

If the \gls{m/z} of an ion is within the \gls{m/z} range of a mass analyzer, 
	then the mass spectrometer that uses this mass analyzer can detect this ion.
Otherwise, this mass spectrometer cannot detect this ion.

A peak is an elevation of intensities within a small interval of \gls{m/z} in a mass spectrum.
Peak shape is such intensity as a function of the \gls{m/z} in this peak. 
A peak is usually bell-shaped (\(\bell\)-shaped).

Resolution is the ability to distinguish between two peaks respectively having two slightly different values of \gls{m/z}.
The IUPAC definition of resolution and the resolving-power definition of resolution coexist.
Similarly, the IUPAC definition of resolving power the and resolving-power definition of resolving power coexist.
Let \(M\) be the \gls{m/z} range of a mass analyzer. 
let \(\Delta{}M\) be the slight difference between the \gls{m/z} of a peak and the \gls{m/z} of another peak.
Let us suppose that these peaks are both produced by a mass spectrometer that uses this mass analyzer. 
According to the IUPAC definition, resolution is defined as \(\frac{M}{\Delta{}M}\), and resolving power is defined as \(\Delta{}M\) \cite{mcnaught1997compendium}.
According to the resolving-power definition, resolution is defined as \(\Delta{}M\), and resolving power is defined as \(\frac{M}{\Delta{}M}\).
The unit of \(\frac{M}{\Delta{}M}\) is none, and the unit of \(\Delta{}M\) is the unit of \gls{m/z}.
Thus, the unit of resolution or of resolving power can indicate which of these two definitions is used. 
Similarly, the peak-width definition of \(\Delta{}m\) and the valley definition of \(\Delta{}m\) coexist. 
According to the peak-width definition, an \(x\%\)-peak-width \(\Delta{}m\) is defined as the width of a peak measured at \(x\%\) of the height of this peak.
The overlap between two \(\bell\)-shaped peaks having the same shape but slightly different \gls{m/z} values produces a \({\bell}{\bell}\)-shaped envelope.
Let us suppose that the minimum height of the valley at the middle of this \({\bell}{\bell}\)-shaped envelope is \(x\%\) of the height of these two \(\bell\)-shaped hills.
According to the valley definition, an \(x\%\)-valley \(\Delta{}m\) is the difference between the \gls{m/z} values of these two \(\bell\)-shaped peaks.

Accuracy is the ability to produce a peak that is overall near the theoretical \gls{m/z} of this peak. 
Let \({p_O}\) be the average \gls{m/z} of the observed peak.
Let \({p_E}\) be the theoretical \gls{m/z} of the expected peak.
Accuracy is defined as either \(\frac{\lvert{p_O} - {p_E}\rvert}{p_E}\) or \(\lvert{p_O} - {p_E}\rvert\). 
Thus, the unit of accuracy can indicate which of these two definitions of accuracy is used.

% very good reference
% https://www.princeton.edu/chemistry/macmillan/group-meetings/SL-mass%20spect.pdf

\subsection{Mass detector} 
\label{subsec:fund2:MS:detector}

Every mass detector records the current-or-charge produced by an ion when this ion hits or passes by a surface.
The working mechanism of an electron-multiplier detector is as follows.
First, an incident ion can cause the ejection of some electrons.
Then, each of these ejected electrons can cause a new ejection of multiple more electrons.
Afterwards, this electron-ejection amplification continues until a huge quantity of electrons produce a detectable signal.
The working mechanism of a Scintillator detector is as follows.
First, an incident ion can cause the emission of some electrons.
Then, this emission of electrons can cause the emission of light, 
Afterwards, this emission of light is detected.
The working mechanism of a Faraday-Cup detector is as follows.          
First, an incident ion collides with a metal, and this collision can cause the ejection of secondary electrons.
Then, this ejection can generate a flow of electric current. %until being recaptured, 
Afterwards, this flow of electric current is detected.
Almost every mass detector amplifies the signal generated by an incident ion.

\section{\texorpdfstring{\Glsfirst{MS/MS}}{MS/MS}} % http://en.wikipedia.org/wiki/Tandem_mass_spectrometry
\label{sec:MS:MSMS}
 
\Glsfirst{MS/MS} is an analytical-chemistry technique that can fragment a biomolecule.
This fragmentation can partially or fully reveal the chemical structure of this fragmented biomolecule.
\Gls{MS/MS} has two stages.
\Gls{MS1} is the first  stage of \gls{MS/MS}.
\Gls{MS2} is the second stage of \gls{MS/MS}.
\Gls{MS2} is immediately after \gls{MS1}.
\Gls{MS1} proceeds as follows.
First, an ionization source ionizes some sample molecules, and these ionized sample molecules are referred to as precursor ions.
Then, a mass analyzer separates these precursor ions based on the \gls{m/z} of these precursor ions. 
Finally, a mass detector detects the \gls{m/z} of some of these separated precursor ions.
\Gls{MS2} proceeds as follows. 
First, an ion filter selects some \gls{MS1}-generated precursor ions that are within a chosen \gls{m/z} range. %GRAMMAR: the precursor ions
Next, some of these selected precursor ions are fragmented. 
Then, some of these fragments become product ions. %GRAMMAR: and become
Afterwards, a mass analyzer separates these product ions based on the \gls{m/z} of these product ions.
Finally, a mass detector detects the \gls{m/z} of some of these separated product ions.
An \gls{MS1} spectrum is defined as the mass spectrum produced in \gls{MS1}.
An \gls{MS2} spectrum is defined as the mass spectrum produced in \gls{MS2}.
\Gls{MS1} spectrum, precursor spectrum, survey spectrum, \gls{MS1} scan, precursor scan, and survey scan are all equivalent in meaning.
\gls{MS2} spectrum, \gls{MS/MS} spectrum, product spectrum, \gls{MS2} scan, \gls{MS/MS} scan, and product scan are all equivalent in meaning.
\begin{figure}
\myframe{\includegraphics[width=\textwidth, page=126, clip=true, bb=59  485 417 630]{img/eidhammer2008computational.pdf}}
\caption[
	A schematic of \gls{MS/MS} \cite{eidhammer2008computational}.]{
	A schematic of \gls{MS/MS} \cite{eidhammer2008computational}.
	\Cref{MS2_pept_id} is an example that shows peaks produced by protein \gls{MS}.
}
\label{schematics_MSMS_instrument}
\end{figure}


{
\def \AB{\textnormal{AB}}
\def \A{\textnormal{A}}
\def \B{\textnormal{B}}
\def \M{\textnormal{M}}
\def \H{\textnormal{H}}
The distribution of the product ions generated by \gls{MS/MS} depends on the fragmentation method used for this \gls{MS/MS}.
\Gls{CID}, also known as collision activated dissociation (CAD), is the most popular fragmentation method.
The mechanism of \gls{CID} is as follows. 
First, an electromagnetic field accelerates a precursor-ion \AB{}.
Next, this precursor ion can collide with at least one neutral gaseous molecule \M{}. 
Then, this collision can cause this precursor ion to fragment.
Afterwards, this precursor ion can fragment into one product ion \A{} and one uncharged molecule \B{}. 
Finally, \A{} can be detected. 
%Usually, \M{} is a noble gas. 
\begin{figure}
\begin{tabular}{|c|c|}
\hline & \\
\begin{subfigure}{0.51\textwidth}
\myframe{\includegraphics[height=.72\textheight, page=20,  clip=true, bb=200 142 400 472]{img/banerjee2012electrospray.pdf}}
\caption[]{
	The notation for the major product ions. %: \{\texttt{a}, \texttt{b}, \texttt{c}, \texttt{x}, \texttt{y}, \texttt{z}\}.
}
\label{fig:fund2:listof-product-ions}
\end{subfigure}&
\begin{subfigure}{0.45\textwidth}
\myframe{\includegraphics[height=.72\textheight, page=22,  clip=true, bb=200 341 400 717]{img/banerjee2012electrospray.pdf}}
\caption[]{
	How b-ions and y-ions are formed.
}
\label{fig:fund2:yb-ions-generation-mechanism}
\end{subfigure}
\\\hline
\end{tabular}
\caption[
	Peptide fragmentation in \gls{MS2} \cite{banerjee2012electrospray}.]{
	Peptide fragmentation in \gls{MS2} \cite{banerjee2012electrospray}.
	%By breaking one of three types of bonds in a peptide backbone, 
	%The following six types of product ions can form: \{\texttt{a}, \texttt{b}, \texttt{c}, \texttt{x}, \texttt{y}, \texttt{z}\}.
}
\end{figure}
The following chemical equation describes the mechanism of \gls{CID}.
%The mechanism of \gls{CID} is summarized in the following chemical equation.
\[\AB^+ + \M \to \A^+ + \B + \M\]	
%In electron capture dissociation (ECD), the precursor cation captures a free electron and then fragments. 
%The following chemical equation describes the mechanism of ECD \cite{cooper2005role}.
%\[ [\M + n\H]^{n+} + e^{-} \to \big[ [\M + n\H]^{(n-1)+} \big]^{*} \to \text{fragments} \] 
%Electron transfer dissociation (ETD) is similar to ECD, except that the precursor cation captures the electron of another anion instead of capturing a free electron, 
%In electron transfer dissociation (ETD), the precursor cation captures the electron of an anion and then fragments.
%The following chemical equation describes the mechanism of ETD \cite{syka2004peptide}.
%\[ [\M + n\H]^{n+} + \A^{-} \to \big[ [\M + n\H]^{(n-1)+} \big]^{*} + \A \to \text{fragments} \] 
%Negative electron transfer dissociation (NETD) is similar to ETD, except that the precursor is an anion instead of a cation.
%In negative electron transfer dissociation (NETD), the precursor anion loses one electron to a cation and then fragments.
%the precursor anion dissociates after giving one of its electrons to a cation, 
%		as shown in the chemical equation below \cite{Coon2005880}.
%The following chemical equation describes the mechanism of NETD \cite{Coon2005880}.
%\[ [\M + n\H]^{n-} + \A^{+} \to \big[ [\M + n\H]^{(n-1)-} \big]^{*} + \A \to \text{fragments} \] 
Even if no fragmentation method is used, precursors can still fragment if the energy in these precursors is sufficiently high. 
%Nowadays, \gls{CID} is the most commonly used fragmentation method for \gls{MS/MS}.
}

\Gls{MS2} can break the backbone of a peptide to produce peptide fragments. % α-C, carbonyl C, N
%More specifically, 
%	the bond between alpha-carbon and carbonyl-carbon, 
%	the bond between carbonyl-carbon and nitrogen, and 
%	the bond between nitrogen and alpha-carbon can be broken. 
Breakage of the bond between alpha-carbon    and carbonyl-carbon can generate either an a-ion or an x-ion (\cref{fig:fund2:listof-product-ions}).
Breakage of the bond between carbonyl-carbon and nitrogen        can generate either a b-ion or a y-ion (\cref{fig:fund2:listof-product-ions}).
Breakage of the bond between nitrogen        and alpha-carbon    can generate either a c-ion or a z-ion (\cref{fig:fund2:listof-product-ions}).
Only a positively charged fragment containing the N-terminus of a precursor peptide can become an a-ion, a b-ion, or a c-ion.
Only a positively charged fragment containing the C-terminus of a precursor peptide can become an x-ion, a y-ion, or a z-ion. 
Sufficiently high collision energy in \gls{CID} can even fragment the side chain of a residue. 
Then, this fragmentation can generate some product ions that are not even shown in \cref{fig:fund2:listof-product-ions}.
However, \gls{CID} generates mostly b-ions and y-ions. 
    
%--------------------------------------------------------------------- more stuff on spectrum

\section{Preprocessing of mass spectra} 
\label{sec:MS:prep}

In a mass spectra, the ion intensity at a precise \gls{m/z} is the strength of the signal that is generated by some ions having this \gls{m/z}.
This strength is often the number of such ions that are detected.
A peak is defined as the ion intensities within a small interval of \gls{m/z} in a mass spectrum.
Presumably, some physically and chemically identical ions generate most of the ion intensities in this peak.

A peak is not always be bell-shaped.
The intensity of a peak is the sum of all intensities in this peak. 
The centroid of a peak is characterized by a combination of the representative \gls{m/z} of this peak and the intensity of this peak.
A centroid \(C\) of a peak is mathematically defined as follows \cite{urban2014fundamental}.
\[C \isdefinedas %(\bar{m}, \Sigma{y}) = 
\left(\frac{\displaystyle\sum_{a<m<b} y(m){\cdot}m}{\displaystyle\sum_{a<m<b} y(m)}, \sum_{a<m<b} y(m) \right)\]
	where \(m\) is \gls{m/z}, 
	\(y(m)\) is the ion intensity at \(m\), 
	\(\sum_{a<m<b} y(m)\) is the intensity of this peak, 
	\(a\) is the lower \gls{m/z} border of this peak, 
	and where \(b\) is the upper \gls{m/z} border of this peak.
\cite{urban2014fundamental} presents several procedures for determining both \(a\) and \(b\). 
Centroiding is defined as the process of replacing a peak by the centroid of this peak.
Centroiding transforms a peak into one ion intensity at one \gls{m/z}.
Thus, centroiding reduces, in a mass spectrum, the number of pairs of \gls{m/z} and ion intensity.
Thus, centroiding compresses a mass spectrum although this compression is lossy.
Moreover, centroiding partially removes the noise in a mass spectrum.
Thus, centroiding facilitates the analysis of a mass spectrum.

Isotopes are defined as atoms having the same number of protons but pairwise different number of neutrons.
Thus, isotopes have the same chemical properties but pairwise different masses.
Each of these pairwise differences is a multiple of the mass of a neutron, and the mass of a neutron is approximately \(1.009\si{\dalton}\).
Almost every atom has multiple isotopes.
Isotopic molecules have the same structural formula but pairwise different masses.
This mass difference exists because at least one atom in this structural formula has isotopes.
Isotopes can also refer to isotopic molecules. 
Thus, all isotopes of every molecule pairwise differ in molecular weight.
Each of these pairwise differences is approximately a multiple of \(1.009\si{\dalton}\).
%the molecular weight of every isotope differs from the molecule weight of every other isotope by a multiple of approximately \(1.009\si{\dalton}\).

The mass of a molecule is the sum of the respective masses of the individually distinguishable atoms that collectively constitute this molecule.
The monoisotopic mass of an atom is defined as the mass of the most abundant isotope of this atom.
Nominal mass                     is defined as the monoisotopic mass rounded to the nearest integer. 
The average      mass of an atom is defined as the average of the respective masses of all isotopes of this atom 
		such that this average is weighted by the natural abundance of each of these isotopes.
\Cref{tab:fund2:listof-residues-emph-mass} 
		shows the respective monoisotopic masses of the commonly observed amino-acid residues and the respective average masses of these residues.
\begin{table}
\centering{
	\includegraphics[trim=0cm 6cm 0cm 0cm, width=\textwidth]{img/AATable.pdf}
%\includegraphics[page=2,bb = 20 165 410 580,clip=true,width=\textwidth]{images/2013-massref-web.pdf}
}
\caption[
	Some properties of the commonly observed amino acids \cite{mascot2014amino}.]{
	Some properties of the commonly observed amino acids \cite{mascot2014amino}.
	\label{tab:fund2:listof-residues-emph-mass}
}
\end{table}

In every mass spectrum, the charge state (\(z\)) of a peak is defined as the charge of the ion that generated this peak.
Let us suppose that two ions have the same \(z\) and differ by a mass difference of \(\Delta{}m\).
Then, in every mass spectrum, the respective \gls{m/z} of the peaks respectively generated by these two ions differ by \(\frac{\Delta{}m}{z}\).
Thus, \(n\) isotopes having the same \(z\) and ordered by mass can respectively generate \(n\) isotopic peaks ordered by \gls{m/z}.
Every two consecutive peaks in these isotopic peaks differ by approximately \(\frac{1.009}{z}\) in \gls{m/z}.
The charge-state determination of some isotopic peaks is defined as the process of inferring the \(z\) of these peaks. 
%given that these peaks are presumably generated by multiple corresponding isotopic ions.
Deisotoping of some isotopic peaks is defined as the process of converting these peaks into one representative peak. 
	%\gls{m/z}, %of \(\frac{m}{z}\),
The mass in the \gls{m/z} of this representative peak is usually the monoisotopic mass of the ions that respectively generated these isotopic peaks.
The \(z\) of this representative peak is the common \(z\) of these isotopic peaks.

Let \(A_1\) and \(A_2\) be two chemically and physically identical molecules.
The following can happen: \(A_1\) gains one positive charge to become \(A_1^+\), and \(A_2\) gains two positive charges to become \(A_2^{++}\).
Let \(m_1\) the \gls{m/z} of the peak generated by \(A_1^+\), and let \(m_2\) the \gls{m/z} of the peak generated by \(A_2^{++}\)
Then, \(m_1 + 1.007 \approx 2\cdot m_2\).
In general, some chemically and physically identical molecules can respectively form different ions during ionization.
Then, these different ions respectively generated different peaks having pairwise different \gls{m/z}.
%During ionization, these identical molecules may gain or lose different number of protons and/or electrons.
%After ionization, these identical molecules can respectively have different charges. 
%The generate differently charged precursor ions. 
%These different precursor ions respectively generate different peaks having different \gls{m/z}.
Deconvolution is defined as the process of converting these different peaks into one peak by assuming the following.
%Instead of gaining or losing multiple protons or multiple electrons, each sample molecule gains or loses one single proton or one single electron.
After ionization, each sample molecule can only gain exactly one proton (electron) instead of being able to gaining multiple protons (electrons).
%gain one electron, loses one proton, or loses one electron.

During fragmentation in \gls{MS2}, an ion might lose part of this ion such that the \(z\) of this ion remains unchanged.
This lost part is usually a small molecule, such as \(\text{H}_2\text{O}\) or \(\text{N}\text{H}_3\).
Such loss is referred to as neutral loss. 

{
\def \m {c}
\def \M {C}
%\def \MW {\ensuremath{\textnormal{mm}^\textnormal{+n}}}
\def \mMass {m} 
\def \setofMass {M}

The mass of a proton  is approximately \(1.007\si{\dalton}\).
The mass of a neutron is approximately \(1.009\si{\dalton}\).
Let \(\M\) be any collection of chemically identical molecules. 
Let \(\mMass\) be the monoisotopic mass, in \(\si{\dalton}\), of \(\M\).
%		%this that a constant number of elementary charge(s) is on any of this given cations.{}
%Let \(\mMass\) be the monoisotopic mass of \(\M\) in \(\si{\dalton}\).
%%If each \(\m\in\M{}\) is singly charged, then each \(\m\in\M\) can only generate a peak at one of the following \gls{m/z}.
Let us suppose that each isotope of \(\M\) gains one proton to become a cation.
Then, an approximate \gls{m/z} of the peak generated by this cation is in the following set:
		\begin{tightcenter}
		\{
		\(\dots, 1.007+\mMass-2{\times}1.009,1.007+\mMass-1{\times}1.009,\) \\
		\(1.007+\mMass,\) \\
		\(1.007+\mMass+1{\times}1.009,1.007+\mMass+2{\times}1.009, \dots\) 
		\}
	\end{tightcenter}
where these different \gls{m/z} correspond to different isotopes.
%If every \(\m\in\M\) can be multiply charged,
%		then every \(\m\in\M\) whose mass is the monoisotopic mass of \(\M\) can only generate the peaks at some of the following \gls{m/z}:
Let us suppose that the most naturally abundant isotope of \(\M\) forms some pairwise different cations.
%Then, each of these cations can generate a peak only at approximately one of the following \gls{m/z}:
Then, an approximate \gls{m/z} of the peak generated by any of these cations is in the following set:
		\begin{tightcenter}
		\{\(\displaystyle
		\frac{\mMass+1{\times}1.007}{1},
		\frac{\mMass+2{\times}1.007}{2},
		\frac{\mMass+3{\times}1.007}{3},\dots
		\)\}
		\end{tightcenter}	
where these different \gls{m/z} correspond to different charge states (\(z\)).
%By definition of charge state, \(z\) is equivalent to the charge state of a peak generated by this cation in a mass spectrum; 
%	then \(\displaystyle\frac{\setofMass+1.007{\times}z}{z}\) is approximately the \gls{m/z} of this peak, 
Let \(\setofMass\) be the set that contains only the respective masses of all isotopes of \(\M\).
Then, for each \(m'\in\setofMass\), there exists an integer \(n\) such that \(m'\approx\mMass + n{\times}1.009\).
Let \(Z\) be the set that contains only the respective \(z\) of all cations formed by \(\M\).
Then, \(Z\subset\{1,2,3,\dots\}\), and \(Z\) is consecutive because ionization efficiency as a function of \(z\) is bell-shaped.
Thus, for each isotope of \(\M\) and for each cation formed by this isotope, the \gls{m/z} of the peak formed by this cation is in the following set.
\[
%\text{The respective \gls{m/z} of the peaks generated by ionization of \(\M\)} \approx 
	\bigcup_{\substack{\textstyle m' \in \setofMass}}\big(\bigcup_{\textstyle z' \in Z}^{}(\frac{m'+1.007\cdot z'}{z'})\big)\] 
%where \(\MW\) is the set that contains only the mass of every isotope of chemically identical cations, 
%	and 
%By definition of isotope, \(\setofMass\) is a sequence of numbers characterized by an increment of approximately 1.009,
%		the approximated mass of a neutron; 
%		%\footnote{A better approximation for the mass of a neutron is 1.008665};
%	moreover, ionization efficiency as a function of charge state is bell-shaped,
%		so \(Z\) is a sequence of natural numbers characterized by an increment of 1.
%By knowing the \gls{m/z} of peaks that are presumably generated by isotopes of chemically identical cations, 
%	we can compute \(\setofMass\) and \(Z\) because \(\setofMass\) and \(Z\) have restricted values \cite{glish2003basics}.
Both \(\setofMass\) and \(Z\) can be observed in a mass spectrum. 

The combination of deconvolution and deisotoping converts all peaks generated by \(\M\) into one single peak.
This single peak is presumably generated by the addition of one proton to the most naturally abundant isotope of \(\M\).
%The relative abundance of each \(\m'\) in \(\M\) should be directly proportional to the relative intensity of the peaks generated by \(\m'\).
%If theoretical peaks match observed peaks in both \gls{m/z} and intensities, 
%	then the combination of such deconvolution and such deisotoping is valid.
%Otherwise, this combination is invalid.
%Then, a match between theoretical peaks and observed peaks in terms of \gls{m/z} and intensities validates such combination of deconvolution and %deisotoping,
%		and the absence of this match invalidates such combination of deconvolution and deisotoping.
}

\section{\texorpdfstring{\Glsfirst{PSM}}{PSM}} 
\label{sec:MS:PSM}
	
In an \gls{MS2} spectrum, the respective masses of some product ions formed by a peptide \M{} can be estimated by the following procedure.
\begin{enumerate}[nolistsep]
\item 
	Determine the mass offset \(\Delta m\) that is specific to the product ion formed by \M{}.
	As shown in \cref{fig:fund2:yb-ions-generation-mechanism}, 
			Both y-ion and b-ion are mostly composed of residues chained together by peptide bonds,
			y-ion possesses one extra \WaterHOH{} and one extra hydrogen atom (H), and
			b-ion possesses one extra H.
	Thus,
			\(\Delta m \approx 19.018\) for y-ion and \(\Delta m \approx 1.008\) for b-ion.
\item 
	Determine the direction in which the residues of \(\M{}\) are iterated. 	
	This direction is from N-terminus to C-terminus for a-ion, b-ion, and c-ion. 
	This direction is from C-terminus to N-terminus for x-ion, y-ion, and z-ion.
\item 
	Iterate the residues of \(\M\). 
	In each iteration, let \(\Sigma m\) be the sum of the respective masses of all residues which are iterated. 
	Add \(\Delta m + \Sigma m\) to the output list of the respective masses of these product ion formed by \M{}.
\end{enumerate}
\Cref{MS2_pept_id} shows example of the application of such procedure.
\begin{figure}
\center
\begin{tabular}{c c}
\myframe{\includegraphics[bb = 70 50 730 487,clip=true, page=25,width=0.48\textwidth]{img/info_Peptide_Fragmentation_Course.pdf}}&
\myframe{\includegraphics[bb = 70 50 730 487,clip=true, page=26,width=0.48\textwidth]{img/info_Peptide_Fragmentation_Course.pdf}} \\
\myframe{\includegraphics[bb = 70 50 730 487,clip=true, page=27,width=0.48\textwidth]{img/info_Peptide_Fragmentation_Course.pdf}}&
\myframe{\includegraphics[bb = 70 50 730 487,clip=true, page=28,width=0.48\textwidth]{img/info_Peptide_Fragmentation_Course.pdf}} \\
\end{tabular}
\caption[An interpretation of an \gls{MS2} spectrum by \gls{PSM}.]
        {An interpretation of an \gls{MS2} spectrum by \gls{PSM}.
         The peptide \texttt{SGFLEEDELK} generated this \gls{MS2} spectrum.
         The peak at the \gls{m/z} of 583.5 is the doubly charged precursor of \texttt{SGFLEEDELK}.
        (Upper left) The theoretical molecular weight (MW) in \(\si{\dalton}\), the notation, and the sequence, of each b-ion of \texttt{SGFLEEDELK} and of each y-ion of \texttt{SGFLEEDELK}.
        %Respective molecular weights (MWs), in nominal mass definition and in \(\si{\dalton}\), of the b-ions and y-ions of \texttt{SGFLEEDELK}.
        (Upper right) An uninterpreted \gls{MS2} spectrum generated by \texttt{SGFLEEDELK}.
        (Lower left)  For some high-intensity peaks, the \gls{m/z} of this peak matches the theoretical MW of an y-ion of \texttt{SGFLEEDELK}.
        (Lower right) For some high-intensity peaks, the \gls{m/z} of this peak matches the theoretical MW of a b-ion of \texttt{SGFLEEDELK}.
}
\label{MS2_pept_id}
\end{figure}
%Peptide identification is the act of identifying the peptides that are in the sample analyzed by a run of \gls{LC-MS/MS}.
%A peptide-identification algorithm takes as input some mass spectra produced by such run and outputs some peptides.
%These peptides presumably generated some of these mass spectra.
%If this algorithm optionally takes as input the protease that cleaved the proteins in the sample analyzed by this run,
%	then these outputted peptides are constrained to be generated by the proteolysis by this inputted protease.
Such procedure can be used for calculating a \gls{PSM} score.
The \gls{PSM} between a peptide and a mass spectrum implies that this peptide is likely to have generated some signals in this mass spectrum.
%A high \gls{PSM} score between a peptide and a mass spectrum implies that this peptide is likely to generate some signals in this mass spectrum. 
Thus, \Gls{PSM} can be used for identifying the peptide that generated some signals in a given mass spectrum.


\textit{De novo} sequencing derives information about the sequence of a presumably novel peptide. 
This sequence presumably has never been discovered before this \textit{de novo} sequencing.
Database search derives information about the sequence of a peptide such that this sequence can be extracted from a database. %TODO: synonym for "extracted" 
Usually, this peptide can be generated by the proteolysis of at least one protein found in this database.
This proteolysis is usually catalyzed by a given protease.
\textit{De novo} sequencing explores more peptide sequences than database search.
Thus, \textit{de novo} sequencing is more error-prone than database search. 
Some commonly used database-search software packages are 
	Mascot \cite{cottrell1999probability}, 
	PEAKS DB \cite{zhang2012peaks}, 
	Sequest \cite{eng1994approach}, 
	MS-GFDB \cite{kim2010generating}, 
	X!Tandem \cite{craig2004tandem}, 
	and OMSSA \cite{geer2004open}.

\Gls{PTM} is any \textit{in vivo} covalent modification to a protein after this protein has been synthesized. 
\Gls{PTM} is presented in more detail in \cite[Chapter~20]{naik2012essentials}. % Essentials of Biochemistry By Naik Chapter 20
Almost every \gls{PTM} can be detected in a mass spectrum as a mass shift.
Thus, \gls{MS} can often identify and sometimes quantitate a \gls{PTM}.
\Gls{FPOP} usually shifts the mass of a biomolecule by \(+15.99\si{\dalton}\) (\cref{tab:AA-OH-reaction-rate}).
However, \gls{FPOP} is not \textit{in vivo}.
Thus, \gls{FPOP} does not generate any \gls{PTM}.

\section{\texorpdfstring{\Glsfirst{MS}}{MS} protocols}
\label{sec:fund2:MS-protocols}

\Cref{fig:fund2:wetlab-workflow} shows some workflows of protein \gls{MS}.
%Bottom-up proteomics is the most common workflow.
\Cref	{fig:fund2:RPMS-schematics} is an schematic of a typical \gls{RP-MS} experiment.
\Gls{RP-MS} is commonly used for studying protein-folding dynamics.
Unfortunately, the spatial resolution of \gls{RP-MS} is only at peptide level. 
This peptide level is an intrinsic characteristic of proteolysis.
Fortunately, by using a variant of the standard \gls{RP-MS}, we can improve the spatial resolution of \gls{RP-MS} to subpeptide level.
\Cref{chap:oxlvl} presents this improvement.

\begin{figure}
The following is the workflow of \gls{LC-MS/MS} for an investigated protein.
\begin{enumerate}[nolistsep,label={\arabic*}]
\item A protease, such as trypsin, cleaves the investigated protein into peptides.
\item \Gls{HPLC} elutes these peptides. Each of these peptides exits the \gls{HPLC} column at the \gls{RT} of this peptide in this column.
\item While \gls{HPLC} is eluting, the mass spectrometer repeats the following procedure.
\begin{enumerate}[nolistsep,label={\arabic*}]
	\item The inlet of the mass spectrometer extracts the eluted peptides from the exit of the \gls{HPLC} column.
	\item The ion source of the mass spectrometer ionizes these extracted peptides by using a soft ionization technique, such as \gls{ESI}.
	      Ionized peptides can be referred to as precursors.
	\item The mass analyzer of the mass spectrometer separates these precursors based on the respective \gls{m/z} of these precursors.
	\item If the conditions for \gls{MS1} are satisfied, 
		      then the mass detector of the mass spectrometer measures the \gls{m/z}-and-intensity of each of these precursors to produce a raw \gls{MS1} spectrum. 
	\item Otherwise, if the conditions for \gls{MS2} are satisfied, 
		      then the mass spectrometer proceeds as follows. 
	\begin{enumerate}[nolistsep,label={\arabic*}]
	\item The mass analyzer selects these separated precursors such that the respective \gls{m/z} of these selected precursors are within a given range.	
	\item The mass analyzer uses a method, such as \gls{CID}, to fragment these selected precursors. 
	      Some of these fragments respectively become product ions.
	\item The mass detector measures the \gls{m/z}-and-intensity of each of these product ions to produce a raw \gls{MS2} spectrum.
	\end{enumerate}
	\item The computer stores this raw \gls{MS1}-or-\gls{MS2} spectrum.
\end{enumerate}
\item Optionally, a software preprocesses such raw mass spectra.
\item A human expert or a software interprets such raw-or-preprocessed mass spectra. 
\end{enumerate} \vspace{5pt}%~\\
\noindent
The following is the workflow of bottom-up proteomics.
\begin{enumerate}[nolistsep]
\label{bottom_up_proteomics}
\item A protein mixture is prepared from cells or tissues.
\item Proteins of interest are extracted from this protein mixture using a conventional molecular-biology technique, such as 1D gel electrophoresis.
\item \Gls{LC-MS/MS} is performed on these proteins of interest.
\end{enumerate} \vspace{5pt}%~\\
\noindent
The following is the workflow of \gls{RP-MS} for an investigated protein. %\Gls{RP-MS} is shown in \cref{fig:fund2:RPMS-schematics}.
\begin{enumerate}[nolistsep]
\item \Gls{FPOP} is performed on the investigated protein. 
	\Gls{FPOP} is described in \cref{fig:fund2:schematics-FPOP}.
\item \Gls{LC-MS/MS} is performed on the protein modified by \gls{FPOP}. 
\end{enumerate}
\caption[
	Some workflows of protein \gls{MS}.]{
	Some workflows of protein \gls{MS}. 	
	\label{fig:fund2:wetlab-workflow}}
\end{figure}

\begin{figure}
\center{
\begin{tabular}{c|c}
\begin{subfigure}{0.48\textwidth}{
	\center{\includegraphics[page=10, bb=420 50 740 520, clip=true, width=\textwidth
	,height=0.55\textheight
	]{img/FPOP-Mass_Spec_09_L_Jones.pdf}}
}
\caption[]{
	a simple schematic of \gls{RP-MS}. 
	\label{fig:fund2:RPMS-schematics_all}}
\end{subfigure}
&
\begin{subfigure}{0.48\textwidth}{
	\center{\includegraphics[page=11, bb=420 50 740 520, clip=true, width=\textwidth
	,height=0.55\textheight
	]{img/FPOP-Mass_Spec_09_L_Jones.pdf}}
}
\caption[]{
	a schematic of \gls{LC-MS} in \gls{RP-MS}. 
	\label{fig:fund2:RPMS-schematics_LCMS}}
\end{subfigure}
\end{tabular}
}
\caption[
	A schematic of \gls{RP-MS} \cite{jones2009comprehensive}.]{
	A schematic of \gls{RP-MS} \cite{jones2009comprehensive}.
	More details are presented in \cref{fig:fund2:wetlab-workflow}.
	\label{fig:fund2:RPMS-schematics}
}
\end{figure}
