% T I T L E   P A G E
% -------------------
% Last updated May 24, 2011, by Stephen Carr, IST-Client Services
% The title page is counted as page `i' but we need to suppress the
% page number.  We also don't want any headers or footers.
\pagestyle{empty}
\pagenumbering{roman}

\newtoggle{showTitlepage}
\toggletrue{showTitlepage}
\iftoggle{showTitlepage} {

% The contents of the title page are specified in the "titlepage"
% environment.
\begin{titlepage}
        \begin{center}
        \vspace*{1.0cm}
        
        \Huge
	      {\bf Improving Spatial Resolution of and Error Estimation for Radical Probe Mass Spectrometry}
        
        \vspace*{1.0cm}
        
        \normalsize
        by \\
        
        \vspace*{1.0cm}
        	
        \Large
        XiaoFei Zhao \\
        
        \vspace*{3.0cm}
        
        \normalsize
        A thesis \\
        presented to the University of Waterloo \\ 
        in fulfillment of the \\
        thesis requirement for the degree of \\
        Master of Mathematics \\
        in \\
        Computer Science \\
        \vspace*{2.0cm}
        
        Waterloo, Ontario, Canada, 2015 \\
        
        \vspace*{1.0cm}
        
        \copyright\ XiaoFei Zhao 2015 \\
        \end{center}
\end{titlepage}

% The rest of the front pages should contain no headers and be numbered using Roman numerals starting with `ii'
\pagestyle{plain}
\setcounter{page}{2}

\cleardoublepage % Ends the current page and causes all figures and tables that have so far appeared in the input to be printed.
% In a two-sided printing style, it also makes the next page a right-hand (odd-numbered) page, producing a blank page if necessary.
 


% D E C L A R A T I O N   P A G E
% -------------------------------
  % The following is the sample Declaration Page as provided by the GSO
  % December 13th, 2006.  It is designed for an electronic thesis.
  \noindent
I hereby declare that I am the sole author of this thesis. This is a true copy of the thesis, including any required final revisions, as accepted by my examiners.

  \bigskip
  
  \noindent
I understand that my thesis may be made electronically available to the public.

\cleardoublepage
%\newpage

% A B S T R A C T
% ---------------
\begin{center}\textbf{Abstract}\end{center}

% -> Why did I do it
The function of a protein depends on the structure of this protein.
A commonly used analytical technique for studying protein structure is \gls{RP-MS}.
\Gls{RP-MS} oxidizes a protein of interest then quantitates the oxidation on this protein.
Such quantitations can probe the \gls{SASA} of this protein.
This \gls{SASA} can be used for studying the structure of this protein.
Thus, the spatial resolution of such quantitations of oxidation is the spatial resolution at which protein folding can be studied.
% -> What did I do
This thesis proposes a computational method for increasing, by many times, the spatial resolution of such quantitations of oxidation.
% -> How did I do it
Traditional \gls{RP-MS} can already quantitate the oxidation on a peptide of a protein.
\gls{MS/MS}, an analytical technique, can fragment a peptide into the suffixes of this peptide.
Thus, the fraction of such individual suffixes of length \(i\) that are oxidized is the relative frequency that one of the last \(i\) residues of this peptide is oxidized.
Thus, two such suffixes of lengths \(i\) and \(j\), where \(i>j\), correspond to two such frequencies.
Thus, the difference between these two frequencies is the frequency that the oxidation on this peptide is inclusively between the \(i\)\textsuperscript{th}-last and \((j+1)\)\textsuperscript{th}-last residues of this peptide.
The oxidation between these two residues is used by our computational method to quantitate oxidation at subpeptide level.
% -> Prove that it works
Such quantitated oxidation extents match the previously published oxidation rates and are computed from an \gls{MS/MS} dataset. 
The \gls{MS/MS} dataset is produced by a specially designed \gls{RP-MS} experiment.
This \gls{RP-MS} experiment used \gls{MS/MS} that targeted six tryptic peptides of apomyoglobin (\gls{PDB} \texttt{1WLA}).

% -> Why did I do it
However, such quantitations of oxidation are not precise, mostly because random errors exist in such fraction of the suffixes that are oxidized.
Such fraction is a type of \gls{AUCXIC} fraction.
A \gls{AUCXIC} fraction represents, in a sample, the quantity of a type of molecules relative to another type of molecules.
% -> What did I do and how did I do
To estimate random errors in a \gls{AUCXIC} fraction, we made three reasonable assumptions partially justified in the literature.
From these assumptions, we mathematically deduced our empirical formula.
Our empirical formula estimates random errors in a \gls{AUCXIC} fraction that is observed in only one run of mass spectrometry.
% -> Proved that it works
Such estimated random errors match the empirically observed random errors in a test dataset.
The test dataset is generated by three almost repeated runs of \gls{MS/MS}.
% -> Why it is applicable to quantitation of oxidation
To generate the test dataset and the \gls{MS/MS} dataset, the same instrument analyzed, with similar configurations, two similar samples.
Thus, our empirical formula is used for estimating random errors in such quantitation of oxidation in the \gls{MS/MS} dataset.

The throughput of \gls{MS/MS} is lower than the throughput of \gls{MSE} by orders of magnitude.
Unfortunately, we showed that, currently, \gls{MSE} almost certainly cannot improve the spatial resolution of \gls{RP-MS}
	presumably because \gls{MSE} generates too much noise.
		
\cleardoublepage
% A C K N O W L E D G E M E N T S
% -------------------------------
\begin{center}\textbf{Acknowledgements}\end{center}
%I would like to thank all the people who made this possible.
I would like to first express my sincere gratitude to my supervisor, 
	Professor Bin Ma, 
	for finding such an interesting interdisciplinary research project for me,
	for guiding me through my project, 
	and especially for improving my scientific-writing skill.
	
I would like to express my gratitude to
		Professor Lars Konermann 
		for improving my knowledge in Chemistry.

I would like to express my gratitude to
		Siavash Vahidi 
		for performing some important wet laboratories,
	because the foundation of my project is based on these wet laboratories. 

I would like to thank my committee members, 
		Professor Ming Li and Professor Forbes Burkowski, 
		for reading and examining this thesis.

I would like to thank all other members of the Bioinformatics research group at the University of Waterloo
		(Dr Hao Lin, Dr Lin He, Dr XueFeng Cui, Dr Laleh Soltan Ghoraie, ShiWei Li, XiaoBo Li, and Eric Marinier) 
		for having some interesting discussions about my project.

I would like to thank
		the University of Waterloo
		for providing a world-class graduate-level education to me.

%I would like to thank my alma mater,
%	McGill University,
%	from which I received my Bachelor of Science in Computer Science and Biology,
%	for the invaluable background knowledge in Computer Science and Biology that I acquired while attending it,
%	which implicitly helped me in my project.

%I would like to thank Xerox Research Center Europe,
%	at which I did an internship in Machine Learning,
%	for the invaluable knowledge and experience in Machine Learning and Mathematical Modeling that I gained while working for it,
%	which ultimately helped me in my project. 

\cleardoublepage
\begin{center}\textbf{Dedication}\end{center}
%This is dedicated to the one I love.
This thesis is dedicated to my parents for the unconditional love that they gave to me.
\cleardoublepage

}{}

% T A B L E   O F   C O N T E N T S
% ---------------------------------
\renewcommand\contentsname{Table of Contents}
\tableofcontents
\cleardoublepage
\phantomsection
% L I S T   O F   T A B L E S
% ---------------------------
\addcontentsline{toc}{chapter}{List of Tables}
\listoftables
\cleardoublepage
\phantomsection		% allows hyperref to link to the correct page
% L I S T   O F   F I G U R E S
% -----------------------------
\addcontentsline{toc}{chapter}{List of Figures}
\listoffigures
\cleardoublepage
\phantomsection		% allows hyperref to link to the correct page

% L I S T   O F   S Y M B O L S	
% -----------------------------

% To include a Nomenclature section
%\addcontentsline{toc}{chapter}{\textbf{Nomenclature}}
%\renewcommand{\nomname}{Nomenclature}
 %\printglossary[title=Nomenclature,toctitle=Nomenclature]
 
 %\printglossary[title=List of Symbols,toctitle=List of Symbols]
 %\printglossary[type=s-this-symbs, title=List of Symbols,toctitle=List of Symbols] 
% \printglossary[title=Nomenclature]
 \cleardoublepage
 \phantomsection % allows hyperref to link to the correct page
 \newpage

% Change page numbering back to Arabic numerals
\pagenumbering{arabic}

\glsresetall
