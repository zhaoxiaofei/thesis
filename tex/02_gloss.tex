\newcommand{\R}[1]{} % has relationship with #1
\newcommand{\rlink}[1]{\text{#1}}
\usepackage[sort=standard,acronym,nomain,toc,section,nopostdot]{glossaries}

\def\chemtype{c-rpms-defns}

\SetCustomStyle
%\setacronymstyle{short-long}

\newglossary[md_lg]{c-math-defns}{md_ot}{md_tn}{Definitions in Mathematics}
\newglossary[ms_lg]{c-rpms-defns}{ms_ot}{ms_tn}{Definitions in mass spectrometry}
\newglossary[ts_lg]{s-this-symbs}{ts_ot}{ts_tn}{Definitions specific to this thesis}

\newacronym{FPOP}               {\text{FPOP}}     {fast photochemical oxidation of protein}
\newacronym{LC}                 {\text{LC}}       {liquid chromatography}
  \R{asi}\newacronym{HPLC}      {\text{HPLC}}     {high-performance liquid chromatography}
  \R{asi}\newacronym{RPHPLC}    {\text{RP-HPLC}}  {reverse-phase \gls{HPLC}}
    \R{asi}\newglossaryentry{UPLC}{
      type= \chemtype,
     	name={\text{UPLC}},
      description={ultra-performance liquid chromatography (UPLC), 
	      which is a trademark of Waters Corporation for its high pressure \gls{HPLC}},
      first={ultra-performance liquid chromatography (UPLC)},
      plural={ERROR}, firstplural={ERROR}
    }  	
\newacronym{MS}                 {\text{MS}}       {mass spectrometry}
  \R{uses}\newacronym{ESI}      {\text{ESI}}      {electrospray ionization}
  \R{uses}\newacronym{QTOF}     {\text{QTOF}}     {quadrupole time-of-flight}
  \newglossaryentry{MS/MS}{
  	type=\chemtype,
  	name={\text{MS/MS}},
  	description={denotes ``tandem mass spectrometry''.},
  	user1={Tandem mass spectrometry}
  }
    \R{uses}\newacronym{CID}    {\text{CID}}      {collision-induced dissociation}
    \R{instate}\newacronym{DDA} {\text{DDA}}      {data-dependent acquisition}
    \R{instate}\newacronym{DIA} {\text{DIA}}      {data-independent acquisition}
\newacronym{RP-MS}               {RP-MS}          {radical-probe mass spectrometry}
    \R{asi} %\newacronym{RP-MS/MS}    {\text{RP-MS/MS}} {radical probe tandem mass spectrometry}
    \newglossaryentry{RP-MS/MS}{	
    	type=\chemtype,
    	name=subpeptide-level RP-MS,
    	description={subpeptide-level radical-probe-mass-spectrometry.},
    	first={subpeptide-level radical-probe-mass-spectrometry (subpeptide-level RP-MS)},
    	plural={ERROR},
    	firstplural={ERROR},
    	%user1={Radical probe tandem mass spectrometry}
    }
  \R{has}\newacronym{RT}        {\text{RT}}       {\textnormal{retention time}}
  \R{has}\newacronym{TIC}       {\text{TIC}}      {total     ion chromatogram}
  \R{has}\newacronym{XIC}       {\text{XIC}}      {extracted-ion chromatogram}
\newacronym{SASA}               {\text{SASA}}     {solvent-accessible surface area}
\newacronym{PTM}                {\text{PTM}}      {post-translational modification}

\newglossaryentry{PDB}{
	type=\acronymtype,
	name={\text{PDB}},
	description={Protein Data Bank},
	plural={ERROR},
	first={PDB},
	firstplural={ERROR}	
}
\newglossaryentry{PSM}{
	type=\acronymtype,
	name={\text{PSM}},
	description={
		peptide-spectrum match 
		%which is the observed similarity between the theoretical mass spectrum of a peptide and an observed mass spectrum
		},
	plural={PSMs},
	first={peptide-spectrum match (PSM)},	
	firstplural={peptide-spectrum matches (PSMs)}	
}
\newglossaryentry{m/z}{
	type=\chemtype,%c-chem-defns,
	name={\ensuremath{\text{$m/z$}}},
	description={
		is the mass-to-charge ratio measured in \ensuremath{\si{\dalton}}.} 
		%\ensuremath{\si{\dalton\per\elementarycharge}}. 
%		Dalton (\si{\dalton}) or equivalently atomic mass unit (a.m.u.) is defined as \ensuremath{\frac{1}{12}} of the mass of an unbound neutral atom of carbon-12 in its nuclear and electronic ground state;
%		\ensuremath{\si{\elementarycharge}} is defined as the elementary positive charge
}

\newglossaryentry{OH-rad} {
	type=\chemtype,
	name={\text{HO\textperiodcentered}}, 
	description={is the symbol for hydroxyl radical.},
	plural={\text{HO\textperiodcentered}},
	first={hydroxyl radical ({\text{HO\textperiodcentered}})},
	firstplural={hydroxyl radicals ({\text{HO\textperiodcentered}})},
	symbol={\text{HO\textperiodcentered}}
}
\newglossaryentry{MS1} {
	type=\chemtype, 
	name={\text{MS\ensuremath{^1}}}, 
	description={
		is the first  stage in \gls{MS/MS} or the only stage in \gls{MS}; 
			\gls{MS1} generates precursor ions and survey  scans.}
}
\newglossaryentry{MS2} {
	type=\chemtype,
	name={\text{MS\ensuremath{^2}}}, 
	description={
		is the second stage in \gls{MS/MS}; 
			\gls{MS2} generates product   ions and product scans.}
}
\newglossaryentry{MSE} {
	type=\chemtype, 
	name={\text{MS\textsuperscript{E}}}, 
	description={
		is a mass-spectrometry technology pioneered by Waters Corporation \cite{plumb2006uplc};
		in \gls{MSE}, \gls{CID} alternates between
		low energy mode which produces \gls{MS1}-like spectra and high energy mode which produces \gls{MS2}-like spectra.}
}
\newglossaryentry{_UPLC} {
	type=\chemtype,
	name=\text{ultra-performance liquid chromatography}, 
	description={is a trademark of Waters Corporation for its high pressure \gls{RPHPLC}.}
}
\newglossaryentry{LC-MS} {	
	type=\chemtype,
	name={\text{LC-MS}}, 
	description={
		is an analytical-chemistry technique using \gls{LC} and \gls{MS} 
			such that the outlet of the \gls{LC} column is connected to the inlet of the \gls{MS} instrument.}
}
\newglossaryentry{LC-MS/MS} {	
	type=\chemtype,
	name={\text{LC-MS/MS}}, 
	description={
		is  an analytical-chemistry technique using \gls{LC} and \gls{MS/MS} 
			such that the outlet of the \gls{LC} column is connected to the inlet of the \gls{MS/MS} instrument.
	}
}
\newglossaryentry{UPLC-MS/MS}{
	type=\chemtype,
	name={\text{UPLC-MS/MS}}, 
	description={\acrshort{LC-MS/MS} where the \acrshort{LC} in such \acrshort{LC-MS/MS} is specifically \gls{UPLC}.}
}

\newglossaryentry{IE}{
	type=\acronymtype,
	name={\text{IE}},
	description={\text{iterative exclusion}},
	first={iterative exclusion (IE)}
}
\newglossaryentry{IE-MS}{
	type=\acronymtype,
	name={\text{IE-MS}},
	description=iterative-exclusion \glsfirst{MS},
	first={iterative-exclusion mass spectrometry (IE-MS)}
}

\newglossaryentry{PRM}{
	type=\chemtype,
	name={\text{PRM}},
	description={\text{Parallel reaction monitoring}},
	first={parallel reaction monitoring (PRM)},
	plural={ERROR},
}

\newcommand{\aIon}  {\ensuremath{a}}
\newcommand{\bIon}  {\ensuremath{b}}
\newcommand{\cIon}  {\ensuremath{c}}
\newcommand{\xIon}  {\ensuremath{x}}
\newcommand{\yIon}  {\ensuremath{y}}
\newcommand{\zIon}  {\ensuremath{z}}
\newglossaryentry{ox1}{
	type=\chemtype,
	name={\text{mono-oxidation}},
	description={
		is defined as a modification characterized by a mass shift of approximately \(15.99\si{\dalton}\) to a biomolecule;
		\(15.99\si{\dalton}\) is approximately equal to the mass of one oxygen atom.}
}
\newglossaryentry{mono-oxidized}{
	type=\chemtype,
	name={\text{mono-oxidized}},
	description={
		denotes ``affected by \gls{ox1}''.}
}
\newglossaryentry{ox2}{
	type=\chemtype,
	name={\text{di-oxidation}},
	description={
		is defined as a modification that is characterized by a mass shift of approximately $31.98\si{\dalton}$
		to an amino acid residue.}
}


% ========== ========== ========== ========== ========== NEW DEFINED TERMS

%\newcommand{\OxOne}   {\text{\ensuremath{{\Delta}m{\approx}16} oxidation}}
%\newcommand{\OxTwo}   {\text{\ensuremath{{\Delta}m{\approx}32} oxidation}}

%\newcommand{\A}     {\ensuremath{A}}
%\newcommand{\B}     {\ensuremath{B}}
%\newcommand{\C}     {\ensuremath{C}}
%\newcommand{\X}     {\ensuremath{X}}
%\newcommand{\Y}     {\ensuremath{Y}}
%\newcommand{\Z}     {\ensuremath{Z}}

%\newglossaryentry{msymb:y:pep:to:ion}{
%	type=s-this-symbs,
%	name={\ensuremath{\ddot{\mathrm{y}}}}, 
%	symbol={\ensuremath{\ddot{\mathrm{y}}}}, 
%	description={
%		is the function that is defined below. \newline
%		\(\gls{msymb:y:pep:to:ion}_i^{\mathrm{mod}=n}(\mathrm{P}) \isdefinedas \mathrm{Y}\)
%		with the following specifications.
%		\begin{itemize}[nolistsep]
%		\item \(\mathrm{P}\) is required   to be a type of peptide.
%		\item \(\mathrm{mod}\) is required to be a type of chemical modification, and \(n \in \{0, 1\}\).
%		\item \(\mathrm{Y}\) is guaranteed to be the type of \(\texttt{y}_i\)-ion of \(\mathrm{P}\).
%		\item \(\mathrm{Y}\) is guaranteed to have exactly $n$ modifications of type \(\mathrm{mod}\).
%		\end{itemize}
%		\(\gls{msymb:y:pep:to:ion}_i^{\mathrm{mod}=n}(\mathrm{P})\) represents a particular type of y-ion that \(\mathrm{P}\) can generate}
%}
%\newglossaryentry{msymb:y:ms2_pep:to:AUCXIC}{
%	type=s-this-symbs,
%	name=  {\ensuremath{\ddot{\mathfrak{y}}}}, 
%	symbol={\ensuremath{\ddot{\mathfrak{y}}}},
%	description={
%		is the function that is defined below. \newline
%		\(\gls{msymb:y:ms2_pep:to:AUCXIC}_i^{\mathrm{mod}=n}(s'', P) \isdefinedas 
%			\gls{AUCXIC}\big(s'', (\gls{msymb:y:pep:to:ion}(\mathrm{P}))\big) \) 
%		with the following specifications.
%		\begin{itemize}[nolistsep]
%		\item $s''$ is required to be a collection of \gls{MS2} spectra and
%		\item \(P\) is required to be a type of peptide.
%		\end{itemize}
%		\(\gls{msymb:y:ms2_pep:to:AUCXIC}_i^{\mathrm{mod}=n}(s'', P)\) represents the signal intensity of \(P\) in \(s''\)}
%}
%
%\newglossaryentry{mysymb:y:molecule:to:formationevent}{
%	type=s-this-symbs,
%	symbol={\ensuremath{\ddot{\mathbbm{y}}}},
%	name=\noindent,
%	description={
%		\(\glssymbol{mysymb:y:molecule:to:formationevent}_i(p) \isdefinedas
%			\text{(the event that $p$ generates 
%				$y$ 
%			such that 
%				$y\in\glssymbol{msymb:y:pep:to:ion}_i^{\mathrm{mod}=n}(P)$)}\)
%	where
%		\begin{itemize}[nolistsep]
%		\item \(p\) is required to be exactly one peptide of type $P$.
%		\end{itemize}
%	}
%}

% ========== ========== ========== ========== ========== TECHNICAL TERMS

\newglossaryentry{gg:M}{
		name={\ensuremath{\ddot{\text{M}}}}, 
		symbol={\ensuremath{\ddot{\text{M}}}},
		description={is any type of any collection of molecules}
}

\newglossaryentry{gg:mz}{
	name={}, 
	symbol={mz},
	description={
		\(\glssymbol{gg:mz}(m) \isdefinedas \text{\gls{m/z} value of $m$, given $m\in $}\)
	}
}
\DeclareMathOperator{\funcmz}{\glssymbol{gg:mz}}
\newglossaryentry{f:Mz}{
	name={}, 
	symbol={mz},
	description={
		\(\glssymbol{gg:Mz}(M) \isdefinedas \text{\gls{m/z} value of $M$, given $M \subseteq M$}\)
	}
}
\DeclareMathOperator{\funcMz}{\glssymbol{f:Mz}}

\newglossaryentry{TICIntensity}{
	type=s-this-symbs,
	name={\text{TIC}}, 
	symbol={\text{TIC}},
	description={
	is a function such that \(\gls{TICIntensity}(s)\) is the sum of the respective intensities at all applicable \gls{m/z} values in the mass spectrum \(s\);
	\(\gls{TICIntensity}(s)\) represents the intensity of all ions detected in \(s\).  
	}
}
\newglossaryentry{XICIntensity}{
	type=s-this-symbs,
	name={\text{XIC}}, 
	symbol={\text{XIC}},
	description={
	is a function that outputs the absolute intensity of some investigated molecules in a mass spectrum;
		\(\gls{XICIntensity}(M, s)\) is the sum of the respective intensities of the peaks generated by \(M\) in \(s\),
			given that \(s\) is a mass spectrum,
			and that \(M\) is some investigated molecules.}
			%and that \(M\) is a conceptual molecule which denotes s set of chemically identical and concrete molecules.}
}
\newglossaryentry{AUCXIC}{
	type=s-this-symbs,
	name={\text{peak-area}}, 
	symbol={\text{peak-area}},
	description={
	is a function that outputs the area under the curve of an \gls{XIC};
%	is the definite integral of the \gls{XIC} of a collection of molecules,
%		where the domain of such integral is the \gls{RT} of such collection of molecules
%		and the unit of the y-axis of such \gls{XIC} is ion count or equivalently absolute intensity. 
	let \(M\) be a class of molecules, %, a conceptual molecule which denotes a set of chemically identical and concrete molecules,
		let \(r\) be a set of mass spectra generated by one run of \gls{LC-MS} or of \gls{LC-MS/MS};
	%let \(x\) be the \glstext{XIC} of $P$ such that $x$ is extracted from $s$,
	%	let \(r\) be the \glstext{RT} range of \(P\) in \(s\); 
	then,
		\(\glssymbol{AUCXIC}(M, r) \isdefinedas \sum_{s \in r} \gls{XICIntensity}(M, s)\),
		so \(\glssymbol{AUCXIC}(M, r)\) represents the total absolute quantity of \(M\) detected in \(r\). 
	}
}

\newglossaryentry{typeof:ox=1:pep}{
	type=s-this-symbs,
	name={\ensuremath{\boldsymbol{P}'}}, 
	user1={
			is defined as a chemical superspecies of \gls{mono-oxidized} peptides that are chemically identical up to structural isomerism,
			where this isomerism is only due to the fact that any site on any residue can be \gls{mono-oxidized}},
	description={
			\glsentryuseri{typeof:ox=1:pep};
			%which entered a given mass spectrometer such that
			%all $p{\in}\gls{ox=1:pep}$ have the same sequence of residues,
			%every $p{\in}\gls{ox=1:pep}$ is labeled with exactly one \gls{ox1},
			%every $p{\in}\gls{ox=1:pep}$ is free of any covalent modification except such \gls{ox1}, 
			%and for every $p{\in}\gls{ox=1:pep}$ its \gls{ox1} can be located at any site of any of its residues.
		for example, \gls{typeof:ox=1:pep} can be any of the following:
			\{\texttt{F[+16]DK}\},
			\{\texttt{Y[+16]K}, \texttt{Y[+16]K}\},
			\{\texttt{Y[+16]K}, \texttt{YK[+16]}\},
			and \{\texttt{Y[+15.99]K}, \texttt{Y[+16]K}\};
		however, \gls{typeof:ox=1:pep} cannot be any of the following:
			\{\texttt{Y[+16]K}, \texttt{Y[+16]K[+16]}\}, 
			\{\texttt{Y[+16]K}, \texttt{YK}\}, 
			\{\texttt{Y[+32]K}\},
			\{\texttt{Y[+16]K}, \texttt{Y[+16]K[+14]}\},
			and \{\texttt{Y[+16]K}, \texttt{Y[-16]K}\}.}
}	
%\newglossaryentry{typeof:ox=1:pep}{
%	type=s-this-symbs,
%	name={\ensuremath{\dot{P}'}}, 
%	description={is the type of \gls{ox=1:pep}}
%}
\newglossaryentry{typeof:ox=0:pep}{
	type=s-this-symbs,
	name={\ensuremath{\boldsymbol{P}}}, 
	user1={
			is defined as a chemical species of unoxidized peptides},
	description={
		\glsentryuseri{typeof:ox=0:pep};
%		is a collection of all peptides which entered a given mass spectrometer such that
%			all $p{\in}\gls{ox=0:pep}$ have the same sequence of residues
%			and every $p{\in}\gls{ox=0:pep}$ is free of any covalent modification.
	for example, \gls{typeof:ox=0:pep} 
		can be \{\texttt{FDK}, \texttt{FDK}\}, 
		can be \{\texttt{ALELFR}\},
		cannot be     \{\texttt{FDK}, \texttt{FKD}\},
		and cannot be \{\texttt{FDK}, \texttt{F[+16]DK}\}.}
}	 
%\newglossaryentry{typeof:ox=0:pep}{
%	type=s-this-symbs,
%	name={\ensuremath{\dot{P}}}, 
%	description={is the type of \gls{ox=0:pep}}
%}



\newglossaryentry{stat:indep}{
	type=c-math-defns, 
	name={\ensuremath{\perp}}, 
	description={
		is the statistical independence indicator;
		\(A \perp B\) indicates that the random variables \(A\) and \(B\) are statistically independent.}
}
\newglossaryentry{iid}{
	type=c-math-defns,
	name={\ensuremath{\text{iid}}}, 
	description={denotes ``independent and identically distributed''.}
}
\newglossaryentry{E}{
	type=c-math-defns,
	name={\ensuremath{\textnormal{E}}}, 
	description={is the expectation operator; 
		\(\E[X]\) is the expected value of the random variable \(X\) or equivalently the mean of \(X\).}
}
\DeclareMathOperator*{\E}{\gls{E}}
\newglossaryentry{VAR}{
	type=c-math-defns,
	name={\ensuremath{\textnormal{var}}}, 
	description={is the variance operator; 
		\(\VAR[X]\) is the statistical variance of the random variable \(X\).}
}
\DeclareMathOperator*{\VAR}{\gls{VAR}}
%\newglossaryentry{rpois}{
%	type=c-math-defns,
%	name={\ensuremath{\textnormal{Pois}}}, 
%	description={
%		is the random variable generator for the Poisson distribution, 
%			where $\rpois(\lambda)$ has a single parameter $\lambda$}}
%\DeclareMathOperator{\rpois}{\gls{rpois}}
%\newglossaryentry{rbeta}{
%	type=c-math-defns,
%	name={\ensuremath{\textnormal{Beta}}}, 
%	description={is the random variable generator for the Beta distribution,
%			where $\rbeta(\alpha, \beta)$ has shape parameters $\alpha$ and $\beta$}
%}
%\DeclareMathOperator{\rbeta}{\gls{rbeta}}
%\newglossaryentry{rgamma}{
%	type=c-math-defns,
%	name={\ensuremath{\textnormal{Gamma}}}, 
%	description={is the random variable generator for the Gamma distribution,
%			where $\rgamma(k,\theta)$ has shape parameter $k$ and scale parameter $\theta$}
%}
%\DeclareMathOperator{\rgamma}{\gls{rgamma}}
\newglossaryentry{rnorm}{
	type=c-math-defns,
	name={\ensuremath{\mathcal{N}}}, 
	description={is the random-variable generator for the normal distribution.
		\(\rnorm(\mu, \sigma^2)\) has a mean of \(\mu\) and a variance of \(\sigma^2\).}
}
\DeclareMathOperator{\rnorm}{\gls{rnorm}}
%\newglossaryentry{Unif}{
%	type=c-math-defns,
%	name={\ensuremath{\mathcal{U}}}, 
%	description={
%		is the uniform distribution over a set of elements.
%		$\Unif(S)$ outputs an element $s{\in}S$ uniformly at random given a set $S$ of elements.
%		$\Unif^{n}(S)$ outputs a set $S'{\subseteq}S$ of $n$ elements given a set $S$ of elements 
%			such that $S'{\overset{\gls{iid}}{\sim}}\Unif(S)$. 	
%		$\Unif^{n}(S)$ is equivalent to simple random sampling of $n$ elements from $S$ with replacement}
%}
%\DeclareMathOperator{\Unif}{\gls{Unif}}
%\newglossaryentry{suchthat}{
%	type=c-math-defns,
%	name={\ensuremath{\Big\rvert}}, 
%	description={
%		means ``such that''. 
%		$A{\suchthat}f(A)$ is the subset of the set $A$ such that $f(A)$ is true, 
%			where $f$ is an indicator function}}
%\DeclareMathOperator{\suchthat}{\gls{suchthat}}
%\newglossaryentry{best}{
%	type=c-math-defns,
%	name={\ensuremath{\textnormal{best}}}, 
%	description={
%		is a comparator.
%		If $a$ is better than $b$, then $\best(a,b)$ outputs $a$;
%		if $b$ is better than $a$, then $\best(a,b)$ outputs $b$.
%		$\best_{s{\in}S}(s)$ outputs a $s^*{\in}S$ such that $s^*$ is better than any other $s'{\in}S\setminus{s^*}$}
%}
%\DeclareMathOperator*{\best}{\gls{best}}
%\DeclareMathOperator*{\argbest}{arg\,\gls{best}}

\newglossaryentry{isApproximatelyDistributedAs}{
	type=c-math-defns,
	name={\glssymbol{isApproximatelyDistributedAs}}, 
	description={denotes ``is approximately distributed as''.},
		symbol={\ensuremath{\mathrel{\overset{\makebox[0pt]{\mbox{\normalfont\tiny\sffamily app}}}{\sim}}}}
}
\DeclareMathOperator{\isappdistas}{\glssymbol{isApproximatelyDistributedAs}}

%\newglossaryentry{NHPP}{
%	type=c-math-defns, 
%	name={\text{NHPP}}, 
%	description={non-homogeneous Poisson process},
%	first={non-homogeneous Poisson process (NHPP)},
%	plural={NHPPs}, firstplural={non-homogeneous Poisson processes (NHPPs)}
%}

\makeglossaries

%\newacronym{ASA}{\text{ASA}}{Accessible Surface Area}
%\DeclareMathOperator*{\f}{\mathit{f}}
%\DeclareMathOperator*{\g}{\mathit{g}}
%\newglossaryentry{arg}{name={\textnormal{arg}},
%	description={
%	is a selector.
%	$\displaystyle{\arg\,\f}_{s{\in}S}(\g(s))$ outputs $s^{*}{\in}S$ such that $\f_{s{\in}S}(\g(s)) = g(s^*)$,
%	where $\f$ and $\g$ are two functions
%	}
%}